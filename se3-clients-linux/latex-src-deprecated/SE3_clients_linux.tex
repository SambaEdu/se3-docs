\section{Pour les impatients qui veulent tester rapidement}


\subsection{Installation du paquet se3-clients-linux sur le serveur}
\label{installation}

Il faut que votre réseau local dispose d'une connexion Internet.
Pour commencer, il faut préparer votre serveur Samba en
y installant le paquet \verbtexte{se3-clients-linux}.
Pour ce faire :

\begin{itemize}
\item Si votre serveur est sous Lenny, il faut ouvrir une
console en tant que \verbtexte{root} et lancer :
%
\begin{lstlisting}
apt-get update
apt-get install se3-clients-linux
\end{lstlisting}
%
\item Si votre serveur est sous Squeeze, vous pouvez :
\begin{itemize}
\item ou bien faire l'installation comme sur un serveur 
Lenny (en mode console donc);
\item ou bien faire l'installation en passant
par l'interface d'administration Web du serveur via les
menus \verbtexte{Configuration générale} \Vers
\verbtexte{Modules}. Dans le tableau des modules,
le paquet \verbtexte{se3-clients-linux} correspond à
la ligne avec l'intitulé \verbtexte{Support des clients linux}.

\end{itemize}
\end{itemize}

\begin{alerte}
Attention, dans les versions précédentes du paquet, il fallait
éditer le fichier \verbtexte{/etc/apt/sources.list}%
%
\footnote{ou mieux, créer un fichier \verbtexte{/etc/apt/sources.list.d/se3-clients-linux.list}
car le fichier \verbtexte{/etc/apt/sources.list} peut être réédité à votre insu
lors de mises à jour du serveur.}
%
et ajouter une des deux lignes suivantes :
%
\begin{lstlisting}
# Pour un serveur en version Lenny.
deb http://francois-lafont.ac-versailles.fr/debian lenny se3

# Pour un serveur en version Squeeze.
deb http://francois-lafont.ac-versailles.fr/debian squeeze se3
\end{lstlisting}
%
Désormais ce n'est plus nécessaire. Le paquet est
maintenant inclus dans le dépôt
officiel du projet SambaÉdu, il n'est donc plus indispensable d'ajouter un dépôt
supplémentaire. 

\textbf{Mais si vous souhaitez utiliser la toute dernière version disponible du paquet},
alors il faudra dans ce cas utiliser le dépôt \url{http://francois-lafont.ac-versailles.fr}
comme indiqué ci-dessus.
\end{alerte}

L'installation ne fait rien de bien méchant sur votre
serveur. Vous pouvez parfaitement désinstaller le
paquet du serveur afin que celui-ci retrouve très exactement
le même état qu'avant l'installation (voir la section~\ref{desinstallation}
page~\pageref{desinstallation}).
L'installation se borne uniquement à effectuer les tâches suivantes :
%
\begin{itemize}
\item Création d'un nouveau répertoire : le répertoire \verbtexte{/home/netlogon/clients-linux/}.

\item Création d'un partage Samba supplémentaire sur le serveur à travers
le fichier de configuration \verbtexte{/etc/samba/smb\_CIFSFS.conf} : il s'agit du
partage CIFS nommé \verbtexte{netlogon-linux} correspondant au répertoire
\verbtexte{/home/netlogon/clients-linux/} du serveur.

\item Lecture de certains paramètres du serveur afin d'adapter certains scripts
contenus dans le
paquet \verbtexte{se3-clients-linux} à l'environnement
de votre domaine local. En fait, ces fameux paramètres récupérés lors
de l'installation du paquet sont au
nombre de trois :
\begin{enumerate}
\item l'adresse IP du serveur ;
\item le suffixe de base de l'annuaire LDAP ;
\item l'adresse du serveur de temps NTP.
\end{enumerate}
\end{itemize}

\begin{alerte}
Lors de l'installation du paquet, si jamais vous obtenez un message
vous indiquant que le serveur NTP ne semble pas fonctionner, avant
de passer à la suite, vous devez vous rendre sur la console d'administration
Web de votre serveur (dans \verbtexte{Configuration générale}
\Vers \verbtexte{Paramètres serveur}) afin de spécifier l'adresse d'un
serveur de temps qui fonctionne correctement
(chose que l'on peut vérifier ensuite dans la page de diagnostic du serveur).
Une fois le paramétrage effectué il vous suffit de reconfigurer le paquet
en lançant, en tant que \verbtexte{root} sur une console du serveur,
la commande suivante :
%
\begin{lstlisting}
dpkg-reconfigure se3-clients-linux
\end{lstlisting}
%
Si tout se passe bien, vous ne devriez plus obtenir d'avertissement
à propos du serveur NTP.
\end{alerte}

Votre serveur Samba possède donc un nouveau partage CIFS qui, au
passage, ne sera pas visible par les machines clientes sous Windows.
Attention, le nom du partage CIFS n'est pas le même que le nom
du répertoire correspondant dans l'arborescence locale du serveur :
%
\begin{center}
\begin{tabular}{|c|c|c|}\hline
Nom du partage & Chemin réseau & Chemin dans l'arborescence locale du serveur \\\hline
\verbtexte{netlogon-linux} & \verbtexte{//SERVEUR/netlogon-linux} &
\verbtexte{/home/netlogon/clients-linux/}\\\hline
\end{tabular}
\end{center}
%
Au niveau de l'installation du paquet proprement dite, côté serveur,
plus aucune manipulation supplémentaire n'est nécessaire désormais.

Sachez enfin que si, pour une raison ou pour une autre, il vous est nécessaire
de reconfigurer le paquet pour restaurer des droits corrects sur les fichiers,
ou bien pour réadapter les scripts à l'environnement de votre
serveur (parce que par exemple son IP a changé),
il vous suffit de lancer la commande suivante
en tant que \verbtexte{root} sur une console du serveur :
%
\begin{lstlisting}
dpkg-reconfigure se3-clients-linux
\end{lstlisting}









\subsection{Intégration d'un client GNU/Linux}

Le répertoire \verbtexte{/home/netlogon/clients-linux/} de votre serveur contient
un script d'intégration par type de distribution GNU/Linux. 
Par exemple, le script d'intégration
pour des Debian Squeeze se trouve dans le répertoire :
%
\begin{center}
\verbtexte{/home/netlogon/clients-linux/distribs/squeeze/integration/}
\end{center}
%
et il s'appelle \verbtexte{integration\_squeeze.bash}.
Il faudra exécuter l'un de ces scripts,
en tant que \verbtexte{root},
\textbf{en local} sur le client GNU/Linux que vous souhaitez intégrer.

\begin{RQ}
pour copier en local sur un client GNU/Linux le script d'intégration 
qui se trouve sur le serveur, on
pourra utiliser la bonne vieille clé USB des familles,
mais on pourra aussi user et abuser de
la commande \verbtexte{scp} (très pratique) qui permet d'effectuer
très simplement des copies
entre deux machines (sous GNU/Linux) distantes. Par exemple, sur
le terminal d'un client Debian Squeeze, vous pourriez exécuter
les commandes suivantes :
%
\begin{lstlisting}
# Chemin du fichier sur le serveur. Le joker * nous permet simplement 
# d'économiser la saisie de quelques touches sur le clavier (à
# condition d'en saisir suffisamment pour éviter toute ambiguïté 
# sur le nom du fichier).
SOURCE="/home/netlogon/clients-linux/dist*/squ*/int*/int*"

# Répertoire de destination sur le client GNU/Linux en local. Par
# exemple le bureau, histoire de voir apparaître le fichier
# sous nos yeux.
DESTINATION="/home/toto/Bureau/"

# Et enfin la copie du fichier du serveur vers le client GNU/Linux en local.
# Il faudra alors saisir le mot de passe du compte root du serveur.
scp root@IP-SERVEUR:"$SOURCE" "$DESTINATION"
\end{lstlisting}
%
\end{RQ}

\begin{RQ}
si jamais vous avez un doute sur le type de distribution
de votre client GNU/Linux, vous pouvez lancer dans un terminal
la commande suivante (pas forcément en tant que \verbtexte{root}) :
%
\begin{lstlisting}
lsb_release --codename
\end{lstlisting}
%
Le résultat vous affichera le nom de code de la distribution
(\verbtexte{squeeze} ou \verbtexte{precise} etc.) ce
qui vous indiquera le script d'intégration à utiliser.
\end{RQ}

Supposons par exemple que vous avez copié le script d'intégration
\verbtexte{integration\_squeeze.bash} sur une Debian Squeeze et
que celui-ci se trouve sur votre bureau.
Alors, \textbf{en tant que \verbtexte{root}}, vous
pouvez lancer l'intégration ainsi :
%
\begin{lstlisting}
# D'abord, on se place sur le bureau (ici, il s'agit du bureau de toto).
cd /home/toto/Bureau

# Ensuite, on rend le script exécutable.
chmod u+x integration_squeeze.bash

# Enfin, on lance l'intégration.
./integration_squeeze.bash --nom-client="toto-04" --is --ivl --rc
\end{lstlisting}
%
Les explications sur les options se trouvent plus loin dans le
document à la section~\ref{options-integration} 
page~\pageref{options-integration}.
Si tout se passe bien, le client finira par lancer un redémarrage. Une
fois celui-ci terminé, vous devriez être en mesure d'ouvrir une
session avec un compte du domaine (comme le compte \verbtexte{admin}
ou un compte de type professeur ou de type élève).

\begin{alerte}
Il est préférable qu'aucun compte local du client n'ait le même
login qu'un compte du domaine. Or, lorsqu'on installe un client 
GNU/Linux, on est en général amené à créer au moins un compte local
(en plus du compte \verbtexte{root}).
Si cela vous arrive, arrangez-vous pour que le login de ce
compte ne risque pas de rentrer en conflit avec le login d'un
compte du domaine. 
Vous pouvez utiliser \verbtexte{userlocal} comme login par exemple,
ou autre chose\ldots
\end{alerte}



\section{Visite rapide du répertoire clients-linux/ du serveur}
\label{arborescence}

Afin de faire un rapide tour d'horizon du paquet
\verbtexte{se3-clients-linux}, voici ci-dessous un 
schéma du contenu du répertoire
\verbtexte{/home/netlogon/clients-linux/} du serveur.
Les noms des répertoires possèdent un slash à la fin,
sinon il s'agit de fichiers standards. Certains fichiers ou
répertoires, dont vous n'avez pas à vous préoccuper,
ont été omis afin d'alléger le schéma et les explications
qui vont avec. Les fichiers ou répertoires que vous avez
le droit de modifier pour les adapter à vos besoins sont en
\textbf{\textcolor{green}{vert}}. À l'inverse, vous ne devez
pas modifier tous les autres fichiers ou répertoires
%
\footnote{En fait, vous pouvez le faire bien sûr car vous
êtes \verbtexte{root} sur le serveur. Mais les modifications
effectuées sur les fichiers/répertoires qui ne sont pas en vert
sur le schéma ne survivront pas à une réinstallation ou à une mise à jour
du paquet \verbtexte{se3-clients-linux}.}%
.
%
\begin{lstlisting}[emph={logon_perso,skel,unefois,divers},emphstyle={\color{green}\textbf}]
-- clients-linux/
   |-- bin/
   |   |-- connexion_ssh_serveur.bash
   |   |-- logon
   |   |-- logon_perso
   |   `-- reconfigure.bash
   |-- distribs/
   |   |-- precise/
   |   |   |-- integration/
   |   |   |   |-- desintegration_precise.bash
   |   |   |   `-- integration_precise.bash
   |   |   `-- skel/
   |   `-- squeeze/
   |       |-- integration/
   |       |   |-- desintegration_squeeze.bash
   |       |   `-- integration_squeeze.bash
   |       `-- skel/
   |-- divers/
   |-- doc/
   |   `-- LISEZMOI.TXT
   `-- unefois/

\end{lstlisting}
%
Voici quelques commentaires rapides :
%
\begin{itemize}
\item Le répertoire \verbtexte{bin/} contient en premier lieu
le fichier \verbtexte{logon} qui est le script de logon. Ce
script est véritablement le chef d'orchestre de tous les clients
GNU/Linux intégrés au domaine. C'est lui
qui contient les instructions exécutées
systématiquement par les
clients GNU/Linux juste avant l'affichage de la fenêtre de connexion,
au moment de l'ouverture de session et au moment de la fermeture
de session. Ce script de logon sera expliqué à la 
section~\ref{logon-script} page~\pageref{logon-script}.
En principe, vous ne devez pas modifier ce fichier.
En revanche, vous pourrez modifier le fichier \verbtexte{logon\_perso}
juste à côté.
Ce fichier vous permettra d'affiner le comportement du script
\verbtexte{logon} afin de l'adapter à vos besoins.
Vous trouverez toutes les explications nécessaires
dans la section~\ref{personnalisation} page~\pageref{personnalisation}.

Le répertoire \verbtexte{bin/} contient également le fichier
\verbtexte{connexion\_ssh\_serveur.bash}. Il s'agit simplement
d'un petit script exécutable qui, lorsque sous serez connecté(e)
avec le compte \verbtexte{admin} sur un client GNU/Linux et que vous 
double-cliquerez dessus, vous permettra d'ouvrir une connexion SSH
sur le serveur en tant que \verbtexte{root} (autrement dit une 
console à distance sur le serveur en tant que \verbtexte{root}).
C'est une simple commodité. Bien sûr, il vous
sera demandé de fournir le mot de passe du compte \verbtexte{root}
sur le serveur. Pour fermer proprement la connexion SSH, il vous
suffira de taper sur la console la commande \verbtexte{exit}.

Enfin, le répertoire \verbtexte{bin/} contient le fichier
\verbtexte{reconfigure.bash}. Il s'agit d'un fichier exécutable
très pratique qui vous permettra de remettre les droits par défaut
sur l'ensemble des fichiers du paquet \verbtexte{se3-clients-linux}
se trouvant sur le serveur et d'insérer le contenu du fichier 
\verbtexte{logon\_perso} (votre fichier personnel que vous pouvez
modifier afin d'ajuster le comportement des clients GNU/Linux selon vos préférences)
à l'intérieur du fichier \verbtexte{logon} qui est le seul fichier lu par
les clients GNU/Linux. Vous pourrez lancer cet exécutable
à partir du compte \verbtexte{admin} du domaine sur un client GNU/Linux
intégré. Cet exécutable utilise une connexion SSH en tant que
\verbtexte{root} et à chaque fois il faudra donc saisir le mot de
passe \verbtexte{root} du serveur.


\item Le répertoire \verbtexte{distribs/} contient
un sous-répertoire par distribution GNU/Linux prise en
charge par le paquet. Par exemple, dans le sous-répertoire
\verbtexte{squeeze/}, il y a les dossiers suivants :

\begin{itemize}

\item Un dossier 
\verbtexte{integration/} qui contient
notamment le script d'intégration.
C'est ce script qu'il faudra exécuter en tant que \verbtexte{root}
sur chaque client Squeeze que l'on souhaite intégrer au domaine
du serveur. Les options disponibles dans ce scripts sont
décrites dans la section~\ref{options-integration} à la
page~\pageref{options-integration}.
Le script de \og désintégration \fg{} se trouve
également dans ce dossier, mais ce script est copié sur chaque
client GNU/Linux en local au moment de l'intégration. Voir la
section~\ref{desintegration} page~\pageref{desintegration}
pour plus d'explications sur le script de \og désintégration \fg.


\item Un dossier \verbtexte{skel/} qui contient le profil
par défaut (c'est-à-dire le home par défaut) de tous les utilisateurs
du domaine sur la distribution concernée. Si vous voulez
modifier la page d'accueil du navigateur de tous les
utilisateurs du domaine ou bien si vous voulez ajouter
des icônes sur le bureau, c'est dans ce dossier qu'il
faudra faire des modifications. Vous trouverez toutes
les explications nécessaires dans la section~\ref{profils} à la
page~\pageref{profils}.

\end{itemize}

\item Le répertoire \verbtexte{divers/} ne contient
pas grand chose par défaut et vous pourrez a priori y mettre
ce que vous voulez. L'intérêt de ce répertoire
est que, si vous y placez des fichiers (ou des
répertoires), ceux-ci seront accessibles uniquement
par le compte
\verbtexte{root} local de chaque client GNU/Linux et par le
compte \verbtexte{admin} du domaine. 
En particulier, vous aurez accès au contenu du
répertoire \verbtexte{divers/}
à travers le script de logon et à travers les scripts
\og unefois \fg{} (évoqués ci-dessous) 
qui sont tous les deux exécutés par le
compte \verbtexte{root} local de chaque client GNU/Linux.
Vous trouverez
un exemple d'utilisation possible de ce répertoire
dans la section~\ref{imprimante} à la page~\pageref{imprimante}.

\item Le répertoire \verbtexte{doc/} contient un fichier
texte qui vous indiquera l'adresse URL de la documentation
en ligne que vous êtes en train
de lire actuellement (à savoir le fichier \verbtexte{\currfilebase.pdf})
ainsi que l'adresse URL
des sources au format \LaTeX{} de cette documentation.


\item Le répertoire \verbtexte{unefois/} sert à exécuter des scripts
une seule fois sur toute une \og famille \fg{} de clients
GNU/Linux intégrés au domaine. Ce répertoire peut s'avérer utile pour
effectuer des tâches administratives sur les clients GNU/Linux.
Toutes les explications nécessaires sur ce répertoire
se trouvent dans la section~\ref{unefois} page~\pageref{unefois}.


\end{itemize}





\section{Les options des scripts d'intégration}
\label{options-integration}

Les deux scripts d'intégration \verbtexte{integration\_squeeze.bash} et
\verbtexte{integration\_precise.bash}, qui doivent être exécutés en
tant que \verbtexte{root} en local sur chaque client GNU/Linux à intégrer, 
utilisent exactement le même jeu d'options. En voici la liste.



\begin{itemize}
\item L'option \verbtexte{{-}{-}nom-client} ou \verbtexte{{-}{-}nc} :
cette option vous permet de modifier le nom
d'hôte\footnote{Celui qui se trouve dans le fichier \verbtexte{/etc/hostname}.
Ce n'est pas un nom DNS pleinement qualifié.} du client. Si l'option n'est pas
spécifiée, alors le client gardera le nom d'hôte qu'il possède déjà.
Si l'option est spécifiée sans paramètre, alors le script d'intégration
stoppera son exécution pour vous demander de saisir le nom de la machine.
Si l'option est spécifiée avec un paramètre, comme dans :
%
\begin{lstlisting}
./integration_squeeze.bash --nom-client="toto-04"
\end{lstlisting}
%
alors le script ne stoppera pas son exécution et effectuera directement
le changement de nom en prenant comme nom le paramètre fourni (ici
\verbtexte{toto-04}). Les caractères autorisés pour le choix du nom
sont :
%
\begin{itemize}
\item les 26 lettres de l'alphabet en minuscules ou en majuscules, \textbf{sans accents} ;
\item les chiffres ;
\item le \og tiret du 6 \fg{} (-) ;
\item et c'est tout !
\end{itemize}
%
De plus, \textbf{le nom de la machine ne soit pas faire plus de 15 caractères}.

\item L'option \verbtexte{{-}{-}mdp-grub} ou \verbtexte{{-}{-}mg} :
cette option vous permet d'ajouter un mot de passe dès qu'un utilisateur
souhaite éditer un des items du menu Grub au démarrage. 
En effet, en général, sur
un système GNU/Linux fraîchement installé et utilisant Grub comme chargeur de
boot, il est possible de sélectionner un des items du menu Grub et de 
l'éditer en appuyant sur la touche \verbtexte{e} sans devoir saisir le moindre
mot de passe. Cela constitue une faille de sécurité potentielle car, dans ce cas,
l'utilisateur peut très facilement éditer un des item du menu Grub et démarrer
ensuite via cet item modifié
de manière à devenir \verbtexte{root} sur la machine \textbf{sans avoir à saisir
le moindre mot de passe}. Avec l'option \verbtexte{{-}{-}mg}, quand l'utilisateur voudra
éditer un des items du menu Gub, il devra saisir les identifiants suivants :
%
\begin{itemize}
\item login : \verbtexte{admin} ;
\item mot de passe : celui spécifié avec l'option \verbtexte{{-}{-}mg}.
\end{itemize}
%
Si l'option \verbtexte{{-}{-}mg} n'est pas spécifiée, alors la configuration
de Grub est inchangée et a priori la faille de sécurité sera toujours présente.
Si l'option est spécifié sans paramètre, alors le script d'intégration
stoppera son exécution pour vous demander de saisir (deux fois) le futur mot 
de passe Grub (votre saisie ne s'affichera pas à l'écran). Si l'option est
spécifiée avec un paramètre comme dans :
%
\begin{lstlisting}
./integration_squeeze.bash --mdp-grub="1234"
\end{lstlisting}
%
alors le script ne stoppera pas son exécution et effectuera directement
le changement de configuration de Grub en prenant comme mot de passe
le paramètre fourni (ici \verbtexte{1234}).

\item L'option \verbtexte{{-}{-}mdp-root} ou \verbtexte{{-}{-}mr} :
cette option vous permet de modifier le mot de passe du compte
\verbtexte{root}. Si vous ne spécifiez pas cette option, le mot
de passe du compte \verbtexte{root} sera inchangé. Si vous spécifiez
cette option sans paramètre, alors le script d'intégration
stoppera son exécution pour vous demander de saisir (deux fois)
le futur mot de passe du compte \verbtexte{root} (votre saisie
ne s'affichera pas sur l'écran). Si l'option est spécifiée avec 
un paramètre comme dans :
%
\begin{lstlisting}
./integration_squeeze.bash --mdp-root="abcd"
\end{lstlisting}
%
alors le script ne stoppera pas son exécution et effectuera directement
le changement de mot de passe en utilisant la valeur fournie en
paramètre (ici \verbtexte{abcd}).

\item L'option \verbtexte{{-}{-}ignorer-verification-ldap} ou \verbtexte{{-}{-}ivl} :
cette option, qui ne prend aucun paramètre, vous permet de 
continuer l'intégration sans faire de pause après la vérification
LDAP. En effet, lors de l'exécution du script d'intégration, quel que soit
le jeu d'options choisi, une recherche
dans l'annuaire du serveur est effectuée. Le script lancera une recherche
de toutes les entrées dans l'annuaire correspondant à des machines susceptibles
d'avoir un lien avec la machine qui est en train d'exécuter 
le script d'intégration au domaine. Plus précisément la recherche porte sur
toutes les entrées dans l'annuaire correspondant à des machines qui ont :
%
\begin{itemize}
\item même nom que la machine exécutant le script  ;
\item \textbf{ou} même adresse IP que la carte réseau de la machine exécutant le script ;
\item \textbf{ou} même adresse MAC que la carte réseau de la machine exécutant le script.
\end{itemize}
%
Dans tous les cas, le résultat de cette recherche sera affiché. Si vous n'avez
pas spécifié l'option \verbtexte{{-}{-}ivl}, alors le script s'arrêtera à ce
moment là et vous demandera si vous voulez continuer l'intégration. Si par exemple
vous vous apercevez que le nom d'hôte que vous avez choisi pour votre
client GNU/Linux existe déjà dans l'annuaire du serveur, il faudra peut-être
arrêter l'intégration (sauf si le système GNU/Linux est installé en dual boot
avec Windows sur la machine et que le système Windows, lui,
a déjà été intégré au domaine avec ce même nom). Mais si vous avez spécifié
l'option \verbtexte{{-}{-}ivl}, alors après avoir affiché le résultat de
la recherche LDAP, le script continuera automatiquement
l'intégration sans vous demander de confirmation.

\item L'option \verbtexte{{-}{-}installer-samba} ou \verbtexte{{-}{-}is} :
cette option, qui ne prend aucun paramètre, provoquera l'installation de Samba
sur le client GNU/Linux. Si vous ne spécifiez pas cette option, alors
Samba ne sera pas installé sur le client GNU/Linux. Actuellement, il est
conseillé de spécifier cette option. En effet, lorsqu'un client GNU/Linux
essaye de monter un partage Samba du serveur (notamment le partage \verbtexte{homes}),
des scripts sont exécutés en amont côté serveur et le montage ne sera effectué qu'une
fois ces scripts terminés. Or, l'un d'entre eux peut mettre un certain temps
(environ 4 ou 5 secondes) à se terminer si Samba n'est pas installé
sur la machine cliente. Par conséquent, si vous ne spécifiez pas l'option
\verbtexte{{-}{-}is}, vous risquez d'avoir des ouvertures de sessions
un peu lentes (lors du montage des partages Samba).
Donc pour l'instant, utilisez cette option lors de vos
intégrations.

\begin{alerte}
Pour l'instant, il faut utiliser l'option \verbtexte{{-}{-}is}
systématiquement.
\end{alerte}


\item L'option \verbtexte{{-}{-}redemarrer-client} ou \verbtexte{{-}{-}rc} :
cette option permet de lancer automatiquement un redémarrage du client GNU/Linux
à la fin de l'exécution du script d'intégration. Si vous ne spécifiez pas cette option,
il n'y aura pas de redémarrage à la fin de l'exécution du script.
Sachez que le redémarrage après intégration est nécessaire pour avoir un système 
opérationnel. Si les intégrations se déroulent sans erreur sur vos machines
Linux, vous aurez donc tout intérêt à spécifier à chaque fois l'option
\verbtexte{{-}{-}rc}.

\end{itemize}


Précisons enfin que, quel que soit le jeu d'options que vous aurez choisi,
\textbf{aucun enregistrement dans l'annuaire du serveur ne sera effectué par le
script d'intégration}. Par conséquent, si vous souhaitez que votre client GNU/Linux
fraîchement intégré figure dans l'annuaire du serveur, il faudra passer par une
réservation d'adresse IP de la carte réseau du client
via le module DHCP du serveur.


\begin{alerte}
Une fois un client intégré au domaine, évitez de monter un disque
ou un partage dans le répertoire \verbtexte{/mnt/}. En effet, le
répertoire \verbtexte{/mnt/} est utilisé constamment par le client
GNU/Linux (une fois que celui-ci est
intégré au domaine)
pour y effectuer des montages de partages, notamment au moment
de l'ouverture de session d'un utilisateur du domaine, et ce
répertoire est aussi constamment \og nettoyé \fg{}, notamment juste
après une fermeture de session. Afin d'éviter le \og nettoyage \fg{}
intempestif d'un de vos disques ou d'un partage réseau de votre cru,
utilisez un autre répertoire
pour procéder au montage. Utilisez par exemple le répertoire \verbtexte{/media/}
à la place. En fait, utilisez ce que vous voulez sauf \verbtexte{/mnt/}.
\end{alerte}



\section{La \og désintégration \fg}
\label{desintegration}

Une fois un client GNU/Linux intégré au domaine, celui-ci
possédera \textbf{localement} un script
permettant de le faire \og sortir \fg{} du domaine et de lui redonner (quasiment)
son état avant l'intégration. Il s'agit du script :
%
\begin{center}
\verbtexte{/etc/se3/bin/desintegration\_<nom-de-code>.bash} 
\end{center}
%
où vous pouvez remplacer \verbtexte{<nom-de-code>} par \verbtexte{squeeze},
par \verbtexte{precise} etc. Ce script admet une unique option (qui ne
prend pas de paramètre) : il s'agit de l'option 
\verbtexte{{-}{-}redemarrer-client} ou \verbtexte{{-}{-}rc} qui, comme
son nom l'indique, redémarre la machine à la fin du script de \og désintégration \fg{}.
Sans cette option, la machine ne redémarrera pas automatiquement. Tout comme pour
les scripts d'intégration, après \og désintégration \fg{}, un redémarrage est 
nécessaire pour que le système soit opérationnel. Autre point commun : aucune
modification sur l'annuaire du serveur n'est effectuée lors de l'exécution du
script de \og désintégration \fg{}. En particulier, après avoir \og sorti \fg{}
un client GNU/Linux du domaine,
il faudra effacer vous-même toute trace de ce client dans l'annuaire du serveur.






\section{Les partages des utilisateurs}


\subsection{Liste par défaut des partages accessibles suivant le type de compte}

Attention, cette liste (décrite ci-dessous)
est une liste proposée \textbf{par défaut} par le
paquet. Vous verrez plus loin, à la section~\ref{gestion-montage}
page~\pageref{gestion-montage}, que vous pourrez définir vous-même
la liste des partages disponibles en fonction du compte qui 
se connecte, en fonction de son appartenance à tel ou tel
groupe etc. \textbf{Cette liste est donc tout à fait modifiable}.

\begin{alerte}
\begin{center}
\textbf{Avertissement valable uniquement pour ceux qui ont déjà
installé une version $\bm{n}$ du paquet avec
$\bm{n < 1.1}$}
\end{center}

Attention, depuis la version $1.1$ du paquet, la gestion des
partages accessibles se fait exclusivement dans le fichier
\verbtexte{logon\_perso}. Cela a une conséquence importante
si une version antérieure à la version $1.1$ du paquet est déjà
installée sur votre serveur. En effet, lors de la mise à jour 
du paquet vers une version
$\geq 1.1$, plus aucun partage réseau ne devrait être
monté à l'ouverture de session sur vos clients GNU/Linux
et cela pour tout utilisateur du domaine. 

C'est parfaitement normal car,
lors de la mise à jour du paquet, votre fichier 
\verbtexte{logon\_perso} a été conservé et c'est désormais dans
ce fichier que les commandes de montage des partages sont
effectuées. Or, a priori, votre fichier \verbtexte{logon\_perso}
ne contient pas encore ces commandes de montage.

Il est cependant très facile de retrouver le comportement
par défaut (comme décrit ci-dessous)
au niveau du montage des partages réseau à l'ouverture de session.
Sur une console du serveur, en tant que \verbtexte{root}, il
vous suffit de faire :
%
\begin{lstlisting}
# On se place dans le répertoire bin/.
cd /home/netlogon/clients-linux/bin/

# On met dans un coin votre fichier logon_perso en le renommant
# logon_perso.SAVE (si jamais vous n'avez jamais touché à ce fichier
# alors vous pouvez même le supprimer avec la commande rm logon_perso).
mv logon_perso logon_perso.SAVE

# On reconfigure le paquet. L'absence du fichier logon_perso sera
# détectée et vous retrouverez ainsi la version par défaut de ce 
# fichier.
dpkg-reconfigure se3-clients-linux
\end{lstlisting}
%
Vous retrouverez un comportement par défaut dès que les clients GNU/Linux
auront mis à jour leur script de logon local, c'est-à-dire au plus tard
après un redémarrage des clients (en fait, après une simple
fermeture de session, la mise à jour devrait se produire).
\end{alerte}

Voici la liste, par défaut, des partages
accessibles en fonction du type de compte lors d'une session.

\begin{enumerate}
\item \textbf{Un compte élève} aura accès :
\begin{itemize}
\item Au partage \verbtexte{//SERVEUR/homes/Docs/} via deux liens
symboliques. Tous les deux possèdent le même nom : 
\og \verbtexte{Documents de <login> sur le réseau} \fg{}.
L'un se trouve dans le répertoire \verbtexte{/home/<login>/} et l'autre
dans le répertoire \verbtexte{/home/<login>/Bureau/}.

\item Au partage \verbtexte{//SERVEUR/Classes/} via deux liens
symboliques. Tous les deux possèdent le même nom : 
\og \verbtexte{Classes sur le réseau} \fg{}.
L'un se trouve dans le répertoire \verbtexte{/home/<login>/} et l'autre
dans le répertoire \verbtexte{/home/<login>/Bureau/}.
\end{itemize}

\item \textbf{Un compte professeur} aura accès :
\begin{itemize}
\item Aux mêmes partages qu'un compte élève.
\item Mais il aura accès en plus au 
partage \verbtexte{//SERVEUR/Docs/} via deux liens
symboliques. Tous les deux possèdent le même nom : 
\og \verbtexte{Public sur le réseau} \fg{}.
L'un se trouve dans le répertoire \verbtexte{/home/<login>/} et l'autre
dans le répertoire \verbtexte{/home/<login>/Bureau/}.
\end{itemize}

\item \textbf{Le compte \verbtexte{admin}} aura accès :
\begin{itemize}
\item Aux mêmes partages qu'un compte professeur.
\item Mais il aura accès en plus au 
partage \verbtexte{//SERVEUR/admhomes/} via deux liens
symboliques. Tous les deux possèdent le même nom : 
\og \verbtexte{admhomes} \fg{}.
L'un se trouve dans le répertoire \verbtexte{/home/admin/} et l'autre
dans le répertoire \verbtexte{/home/admin/Bureau/}.
\item Et il aura accès en plus au 
partage \verbtexte{//SERVEUR/netlogon-linux/} via deux liens
symboliques. Tous les deux possèdent le même nom : 
\og \verbtexte{clients-linux} \fg{}.
L'un se trouve dans le répertoire \verbtexte{/home/admin/} et l'autre
dans le répertoire \verbtexte{/home/admin/Bureau/}.
\end{itemize}

\end{enumerate}










\subsection{Le lien symbolique clients-linux}

Rien de nouveau donc au niveau des partages disponibles, à part
le partage \verbtexte{netlogon-linux} accessible via le
compte \verbtexte{admin} du domaine à travers le lien symbolique
\verbtexte{clients-linux} situé sur le bureau. Ce lien symbolique vous permet
d'avoir accès, en lecture et en écriture, au répertoire 
\verbtexte{/home/netlogon/clients-linux/} du serveur.
Techniquement, une modification de ce répertoire est aussi
possible via le lien symbolique \verbtexte{admhomes} puisque celui-ci
donne accès à tout le répertoire \verbtexte{/home/} du serveur.

\subsubsection*{Avertissement : toujours reconfigurer les droits
après modifications du contenu du répertoire clients-linux/}
\label{reconfigurer-droits}

\begin{alerte}
Lors de certains paramétrages du paquet \verbtexte{se3-clients-linux}, vous serez
parfois amené(e) à modifier le contenu du répertoire 
\verbtexte{/home/netlogon/clients-linux/} du serveur :
%
\begin{itemize}
\item soit via une console sur le serveur si vous êtes un(e) adepte de la ligne de commandes ;
\item soit via le lien symbolique \verbtexte{clients-linux} situé sur le bureau du compte
\verbtexte{admin} lorsque vous est connecté(e) sur un client GNU/Linux intégré au domaine.
\end{itemize}
%
Dans un cas comme dans l'autre, une fois vos modifications terminées, il faudra \textbf{TOUJOURS}
reconfigurer les droits du paquet \verbtexte{se3-clients-linux} sans quoi vous
risquez ensuite de rencontrer des erreurs incompréhensibles. Pour ce faire il faudra :
%
\begin{itemize}
\item ou bien, \textbf{si vous êtes connecté(e) en mode console sur le serveur},
exécuter en tant que \verbtexte{root} la commande :
\begin{lstlisting}
dpkg-reconfigure se3-clients-linux
\end{lstlisting}
\item ou bien, \textbf{si vous êtes connecté(e) en tant qu'\verbtexte{admin} sur un
client GNU/Linux}, double-cliquer sur le fichier \verbtexte{reconfigure.bash}
accessible en passant par le lien symbolique \verbtexte{clients-linux} sur
le bureau puis par le répertoire \verbtexte{bin/}
(le mot de passe \verbtexte{root} du serveur sera demandé).
\end{itemize}
%
\textbf{Remarque :} en réalité, ces deux procédures ne font pas que reconfigurer les droits
sur les fichiers, elles permettent aussi d'injecter le contenu du fichier
\verbtexte{logon\_perso} dans le fichier \verbtexte{logon}. Ce point sera abordé
dans la section~\ref{personnalisation} page~\pageref{personnalisation}.
\end{alerte}









\section{La gestion des profils}
\label{profils}



\subsection{Une précision à avoir en tête}

Dans cette documentation, on appellera \og profil \fg{} 
le contenu \textbf{ou une copie du contenu}
du home d'un utilisateur (par exemple le profil de \verbtexte{toto} est le contenu du répertoire
\verbtexte{/home/toto/}). Pour bien comprendre le mécanisme des profils, il faut avoir
en tête ces deux éléments :
%
\begin{enumerate}
\item Le serveur Samba offre un partage CIFS dont le chemin réseau
est \verbtexte{//SERVEUR/netlogon-linux/} et qui correspond sur le serveur
au répertoire \verbtexte{/home/netlogon/clients-linux/}.
\item Sur chaque client GNU/Linux intégré au domaine, le répertoire
\verbtexte{/mnt/netlogon/} est un point de montage (en lecture seule)
du partage CIFS \verbtexte{//SERVEUR/netlogon-linux/}.
\end{enumerate}
%
\textbf{Conclusion à bien avoir en tête :} sur un client GNU/Linux
intégré au domaine, visiter le
répertoire local \verbtexte{/mnt/netlogon/} revient en fin de compte
à visiter le répertoire
\verbtexte{/home/netlogon/clients-linux/} du serveur Samba. Dans le
tableau ci-dessous, les trois \og adresses \fg{} suivantes
désignent finalement la même zone de stockage qui se
trouve sur le serveur :
%
\begin{center}
\begin{tabular}{|c|c|c|}\hline
Chemin réseau & Chemin local sur le serveur & Chemin local sur un client GNU/Linux \\\hline
\verbtexte{//SE3/netlogon-linux} & \verbtexte{/home/netlogon/clients-linux/}
& \verbtexte{/mnt/netlogon/} \\\hline
\end{tabular}
\end{center}



\subsection{Les différentes copies du profil par défaut}


Revenons maintenant à nos profils. En fin de compte, pour un client GNU/Linux donné
qui a été intégré au domaine, il existe plusieurs copies du profil par défaut des
utilisateurs. \textbf{Dans le cas des clients sur Debian Squeeze} par exemple, il y a :
%
\begin{enumerate}

\item Le profil par défaut \textbf{distant} qui est unique
et centralisé sur le serveur. Il est
accessible de plusieurs manières. Les trois \og adresses \fg ci-dessous accèdent
toutes à ce même profil par défaut \textbf{distant} :

\begin{itemize}

\item à travers le réseau via le partage CIFS :
\begin{flushright}
\verbtexte{\textcolor{red}{//SERVEUR/netlogon-linux/}distribs/squeeze/skel/}\hspace*{10em}
\end{flushright}

\item sur le serveur directement à l'adresse :
\begin{flushright}
\verbtexte{\textcolor{red}{/home/netlogon/clients-linux/}distribs/squeeze/skel/}\hspace*{10em}
\end{flushright}

\item sur chaque client Squeeze intégré au domaine via le chemin :
\begin{flushright}
\verbtexte{\textcolor{red}{/mnt/netlogon/}distribs/squeeze/skel/}\hspace*{10em}
\end{flushright}

\end{itemize}

\item Le profil par défaut \textbf{local} qui se trouve sur chaque client intégré au domaine
dans le \textbf{répertoire local} \verbtexte{/etc/se3/skel/} (ce répertoire n'est pas
un point de montage, c'est un répertoire local au client GNU/Linux).

\end{enumerate}




\subsection{Le mécanisme des profils}
\label{mecanisme-profils}

Voici comment fonctionne le mécanisme des profils du point de vue d'un client GNU/Linux 
sous Debian Squeeze (sous une autre distribution, c'est exactement la
même chose) :
\begin{enumerate}
\item \textbf{Au moment de l'affichage de la fenêtre de connexion} du système
(c'est-à-dire soit juste après le démarrage
du système ou soit juste après chaque fermeture de session),
le client GNU/Linux va comparer le contenu de deux fichiers :
%
\begin{enumerate}
\item le fichier \verbtexte{/etc/se3/skel/.VERSION} de son profil par défaut \textbf{local}
\item le fichier \verbtexte{/mnt/netlogon/distribs/squeeze/skel/.VERSION} du profil par défaut \textbf{distant}.
\end{enumerate}
%
Si ces deux fichiers ont un contenu totalement identique,
alors le client GNU/Linux ne fait rien car il estime
que son profil par défaut \textbf{local} et le profil par défaut \textbf{distant} 
sont identiques. Si en revanche les deux fichiers ont un contenu différent,
alors le client va modifier son profil par défaut \textbf{local}
afin qu'il soit identique au profil par défaut \textbf{distant}.
Autrement dit, il va synchroniser%
\footnote{Le terme de synchronisation est bien adapté car c'est
justement la commande \verbtexte{rsync} qui est utilisée pour
effectuer cette tâche.}
son profil par défaut \textbf{local} par rapport au
profil par défaut \textbf{distant}.

\item \textbf{Au moment de l'ouverture de session} d'un compte du domaine,
c'est-à-dire juste après une saisie correcte du login et du mot de passe d'un
compte du domaine, appelons ce compte \verbtexte{toto}, le client GNU/Linux va créer
le répertoire (local) vide \verbtexte{/home/toto/} et le remplir 
en y copiant dedans le contenu de son
profil par défaut \textbf{local} (c'est-à-dire le contenu du répertoire \verbtexte{/etc/se3/skel/})
afin de compléter le home de \verbtexte{toto}.

\item \textbf{Au moment de la fermeture de session}, tous les liens
symboliques situés dans \verbtexte{/home/toto/} qui permettent d'atteindre
les différents partages auxquels \verbtexte{toto} peut prétendre sont
supprimés.

\item \textbf{Au moment du prochain affichage de la fenêtre
de connexion}, c'est-à-dire ou bien juste après la fermeture de session de
\verbtexte{toto} s'il n'a pas
choisi d'éteindre le poste client ou bien au prochain démarrage du système, 
\textbf{le répertoire \verbtexte{/home/toto/} est tout simplement effacé}.
\end{enumerate}



\subsection{Exemple de modification du profil par défaut avec Firefox}
\label{modifier-profil}

Du point de vue de l'utilisateur, cette gestion des profils 
est assez contraignante : par exemple notre cher \verbtexte{toto} 
aura beau modifier son profil durant
sa session (changer le fond d'écran, ajouter un lanceur sur le bureau),
après une fermeture puis réouverture de session, il retrouvera
inlassablement le même profil par défaut et toutes ses modifications
auront disparu. De plus, tous les comptes du
domaine (que ce soit les comptes professeur ou les comptes
élève) possèdent exactement le même profil par défaut\footnote{Cette restriction
pourra, dans une certaine mesure, être levée lorsqu'on abordera la personnalisation du
script de logon à la section \ref{personnalisation} page \pageref{personnalisation}.}.
Seule la liste des partages réseau accessibles sera différente d'un compte à l'autre.
Mais ceci étant dit, cette gestion des profils présente
tout de même deux avantages importants :

\begin{enumerate}
\item \textbf{Ouverture de session rapide :} en effet, au moment de l'ouverture
de session d'un compte du domaine, la création du home ne sollicite pas le
réseau puisqu'elle passe par une simple copie locale du contenu de \verbtexte{/etc/se3/skel/}
qui est copié dans \verbtexte{/home/toto/}.

\item \textbf{Modification du profil par défaut (pour tous les
utilisateurs) simple et rapide :} en effet, il devient très facile de modifier
le profil par défaut des utilisateurs, car, si vous avez bien
suivi, c'est le profil par défaut \textbf{distant}
(celui sur le serveur) qui sert de modèle à tous les profils par défaut \textbf{locaux}
des clients GNU/Linux. Une modification du profil par défaut \textbf{distant} accompagnée
d'une modification du fichier \verbtexte{.VERSION} associé sera impactée
sur chaque profil par défaut \textbf{local} de tous les clients GNU/Linux.
\end{enumerate}


Prenons un exemple avec le navigateur Firefox :
vous souhaitez imposer un profil par défaut particulier au niveau 
de Firefox pour tous les utilisateurs du domaine sur
les clients GNU/Linux de type Precise Pangolin.
Pour commencer, vous devez
ouvrir une session sur un client GNU/Linux Precise Pangolin
et lancer Firefox afin de le configurer exactement comme vous
souhaitez qu'il le soit pour tous les utilisateurs (page d'accueil,
proxy, etc).
Une fois le paramétrage effectué,
pensez bien sûr à fermer l'application Firefox. Ensuite, il vous
suffit de suivre la procédure ci-dessous.
Pour la suite, on admettra que la session utilisée pour fabriquer
le profil Firefox par défaut est celle du compte \verbtexte{toto}.
%
\begin{enumerate}
\item Il faut copier le répertoire \verbtexte{/home/toto/.mozilla/}%
%
\footnote{Car c'est ce répertoire qui contient tous les réglages
concernant Firefox que vous avez effectués.}
%
(et tout son contenu bien sûr)
dans le profil par défaut \textbf{distant} du serveur, et cela
tout en veillant à ce que les droits sur la copie soient corrects.
Pour ce faire, vous avez deux méthodes possibles :
%
\begin{itemize}

\item \textbf{Méthode graphique :} vous copiez le répertoire
\verbtexte{/home/toto/.mozilla/} sous une clé USB puis vous fermez la
session de \verbtexte{toto} pour en rouvrir une avec le compte \verbtexte{admin}
du domaine. Ensuite, vous double-cliquez sur le lien symbolique
\verbtexte{clients-linux} qui se trouve sur le bureau puis vous
vous rendez successivement dans \verbtexte{distribs/} \Vers \verbtexte{precise/}
\Vers \verbtexte{skel/} pour enfin, via un glisser-déposer,
copier dans \verbtexte{skel/}
le répertoire \verbtexte{.mozilla/} qui se trouve dans la clé USB
(le dossier \verbtexte{skel/} devra
donc contenir un répertoire \verbtexte{.mozilla/}).

Attention, en général, les répertoires dont le nom commence par un point
sont cachés par défaut et pour qu'ils s'affichent dans l'explorateur
de fichiers il faudra sans doute activer une option du genre
\og \verbtexte{afficher les fichiers cachés} \fg{}.

Enfin, comme vous avez ajouté des fichiers dans le répertoire
\verbtexte{clients-linux/} du serveur, il faut reconfigurer les droits
des fichiers. Pour ce faire, vous double-cliquez sur le lien symbolique
\verbtexte{clients-linux} qui se trouve sur le bureau puis vous
vous rendez dans \verbtexte{bin/} et vous double-cliquez sur
le fichier \verbtexte{reconfigure.bash} (vous devrez saisir le mot de
passe \verbtexte{root} du serveur).


\item \textbf{Méthode via la ligne de commandes :}
sur la session de \verbtexte{toto} restée ouverte, vous ouvrez
un terminal et vous lancez les commandes suivantes :
%
\begin{lstlisting}
# Répertoire du client GNU/Linux à copier sur le serveur.
SOURCE="/home/toto/.mozilla/"

# Destination sur le serveur.
DESTINATION="/home/netlogon/clients-linux/distribs/precise/skel/"

# Copie du répertoire local (et de tout son contenu) vers le serveur.
scp -r "$SOURCE" root@IP-SERVEUR:"$DESTINATION"
\end{lstlisting}
%
À ce stade, le répertoire \verbtexte{.mozilla/} a bien été
copié sur le serveur mais les droits Unix sur la copie ne sont pas encore corrects.
Pour les reconfigurer, il faut exécuter la commande
\og \verbtexte{dpkg-reconfigure se3-clients-linux} \fg{} en tant que
\verbtexte{root} sur le serveur. Là aussi, cela peut se faire
directement du client GNU/Linux, sans bouger, via ssh avec la commande :
%
\begin{lstlisting}
# Avec ssh, en étant sur le client GNU/Linux, on peut exécuter notre commande
# à distance sur le serveur tant que root.
ssh -t root@IP-SERVEUR "dpkg-reconfigure se3-clients-linux"
\end{lstlisting}

\end{itemize}


\item Modifiez le fichier
\verbtexte{.VERSION}%
\footnote{Ne pas oublier cette étape, sans quoi les clients GNU/Linux 
estimeront que le profil par défaut \textbf{distant}
n'a pas été modifié et la mise à jour du profil
par défaut \textbf{local} n'aura pas lieu}
du profil par défaut \textbf{distant}. 
Ce fichier \verbtexte{.VERSION} est un simple fichier texte, vous pouvez le modifier
avec un simple éditeur. S'il contient la chaîne \og 1 \fg{} par exemple, alors
éditez-le et écrivez \og 2 \fg{} à la place. Si vous préférez, vous 
pouvez très bien indiquer la date du moment comme dans \og Le \today~à~15h04 \fg{}.
Le but est simplement, qu'une fois modifié, le fichier \verbtexte{.VERSION} du
serveur possède un contenu différent de chacun des fichiers \verbtexte{.VERSION}
locaux aux machines clientes.
Dans notre exemple, le fichier 
se trouve dans le répertoire 
\verbtexte{/home/netlogon/clients-linux/precise/skel/} du serveur.
Là aussi, deux méthodes s'offrent à vous pour le modifier :

\begin{itemize}

\item \textbf{La méthode graphique} : si ce n'est pas déjà fait,
vous fermez la session de \verbtexte{toto} pour vous connecter sur le
client GNU/Linux avec le compte \verbtexte{admin} du domaine.
Ensuite, vous double-cliquez sur le lien symbolique
\verbtexte{clients-linux} qui se trouve sur le bureau puis vous
vous rendez successivement dans \verbtexte{distribs/} \Vers \verbtexte{precise/}
\Vers \verbtexte{skel/}. Faites en sorte d'activer l'option
\og \verbtexte{afficher les fichiers cachés} \fg{} afin de voir apparaître
le fichier \verbtexte{.VERSION} qui se trouve à l'intérieur
du dossier \verbtexte{skel/}. Éditez ce fichier
afin simplement de modifier son contenu.
Bien sûr, pensez à enregistrer la modification. Pas besoin ici
de reconfigurer les droits car le fait de modifier le contenu du fichier
\verbtexte{.VERSION} ne change pas les droits sur ce fichier
qui, a priori, étaient déjà corrects.

\item \textbf{Méthode via la ligne de commandes :} 
sur la session de \verbtexte{toto} restée ouverte, vous ouvrez
un terminal et vous lancez les commandes suivantes :
%
\begin{lstlisting}
# Le fichier sur le serveur qu'il faut modifier.
CIBLE="/home/netlogon/clients-linux/distribs/precise/skel/.VERSION"

ssh root@IP-SERVEUR "echo Version du 10 janvier 2012 à 15h04 > $CIBLE"
# Maintenant le fichier contient "Version du 10 janvier 2012 à 15h04".
\end{lstlisting} % $

\end{itemize}

\end{enumerate}



Dès le prochain affichage de la fenêtre de connexion,
les profils par défaut \textbf{locaux} de tous les clients Precise Pangolin
seront modifiés afin d'être identiques au profil par défaut \textbf{distant}
du serveur. Dès lors, les utilisateurs bénéficieront des paramétrages
de Firefox que vous avez effectués.

De la même manière que précédemment, sur le profil par défaut \textbf{distant},
vous pouvez parfaitement définir le contenu du bureau des utilisateurs :
au lieu de copier un répertoire \verbtexte{.mozilla/} sur le serveur, ce
sera un répertoire \verbtexte{Bureau/}, mais le principe reste le même.


\begin{alerte}
D'une distribution à une autre, les versions des logiciels n'étant
pas forcément identiques, chaque distribution prise en charge possède
son propre profil par défaut \textbf{distant}. Sur le serveur Samba, on a donc :
\begin{itemize}
\item le répertoire \verbtexte{/home/netlogon/clients-linux/distribs/squeeze/skel/} pour les
Debian Squeeze.
\item le répertoire \verbtexte{/home/netlogon/clients-linux/distribs/precise/skel/} pour les
Ubuntu Precise Pangolin.
\end{itemize}
\end{alerte}

\subsection{Personnaliser le profil en fonction de l'utilisateur}
\label{personnaliser-profil}

La rigidité de la gestion du profil telle qu'elle est décrite à la section~\ref{modifier-profil} 
peut cependant être contournée en modifiant le script de logon\footnote{Le fonctionnement du script 
de logon est décrit dans la section~\ref{logon-script}, page~\pageref{logon-script}.}. Pour comprendre 
cela, poursuivons avec l'exemple de la modification du profil par défaut de Mozilla. Imaginons que vous 
souhaitiez que les enseignants disposent d'un navigateur dont la configuration diffère de celle à laquelle 
accèdent les élèves (extensions particulières, favoris différents, etc.).

Dans ce cas, vous copierez sur le répertoire \verbtexte{/skel} du serveur le répertoire 
\verbtexte{/home/toto/.mozilla} après l'avoir renommé en \verbtexte{/.mozilla-prof}.

Évidemment, dans ce cas, si un enseignant ouvre une session et lance son navigateur, la 
configuration prise en compte par le système sera toujours celle du répertoire \verbtexte{/.mozilla}. 
Il faut donc, pour achever ce processus, modifier le fichier \verbtexte{logon\_perso} pour 
qu'au moment de l'ouverture de session, le répertoire \verbtexte{/.mozilla} soit remplacé 
par \verbtexte{/.mozilla-prof} si et seulement si c'est un(e) enseignant(e) qui se connecte.

Pour ce faire, vous utiliserez les variables prêtes à l'emploi (voir section~\ref{fonctions-utiles}), 
et indiquerez dans la fonction \verbtexte{ouverture\_perso} les lignes suivantes:

\begin{lstlisting}
     # chargement du profil mozilla pour les profs
    if est_dans_liste "$LISTE_GROUPES_LOGIN" "Profs"; then
        rm -rf "$REP_HOME/.mozilla"
        mv "$REP_HOME/.mozilla-prof" "$REP_HOME/.mozilla"
    fi
\end{lstlisting}

Ainsi, à l'ouverture de session, si l'utilisateur qui se connecte est un(e) enseignant(e), 
le \verbtexte{logon\_perso} commencera par supprimer le répertoire \verbtexte{.mozilla}, 
puis renommera \verbtexte{/.mozilla-prof} en \verbtexte{/.mozilla}, permettant ainsi au 
système de prendre en compte ce répertoire pour la configuration du navigateur.

Vous imaginez la suite: on peut, avec cette méthode, personnaliser la configuration de 
tous les logiciels et de l'environnemnet de bureau pour chaque profil, voire pour chaque utilisateur.

\section{Le répertoire unefois/}
\label{unefois}

\subsection{Principe de base}

Si vous souhaitez faire des interventions ponctuelles sur les clients
GNU/Linux sans vous déplacer devant les postes, alors le répertoire 
\verbtexte{/home/netlogon/clients-linux/unefois/} du serveur Samba
peut vous intéresser. En effet, des fichiers exécutables
placés dans ce répertoire seront susceptibles d'être lancés une
seule fois
sur les clients GNU/Linux lors du démarrage.
En pratique, vous allez créer un sous-répertoire à la racine
du répertoire \verbtexte{unefois/} du serveur. Par exemple :
%
\begin{itemize}
\item Si le nom de ce sous-répertoire est \verbtexte{dell740},
alors les exécutables se trouvant dans ce sous-répertoire
seront lancés une fois au démarrage de tous les clients GNU/Linux dont le
nom de machine \textbf{contient à la casse près} 
la chaîne de caractères \verbtexte{dell740}.

\item Si le nom de ce sous-répertoire est \verbtexte{\string^S121-}%
\footnote{Oui, il s'agit bien d'un répertoire dont le nom commence par un accent circonflexe.},
alors les exécutables se trouvant dans ce sous-répertoire
seront lancés une fois au démarrage de tous les clients GNU/Linux dont le
nom de machine \textbf{commence à la casse près par}
la chaîne de caractères \verbtexte{S121-}.

\item Si le nom de ce sous-répertoire est \verbtexte{prof\$},
alors les exécutables se trouvant dans ce sous-répertoire
seront lancés une fois au démarrage de tous les clients GNU/Linux dont le
nom de machine \textbf{se termine à la casse près par}
la chaîne de caractères \verbtexte{prof}.


\item Si le nom de ce sous-répertoire est \verbtexte{\string^S121-HP-P\$},
alors les exécutables se trouvant dans ce sous-répertoire
seront lancés une fois au démarrage du client GNU/Linux dont le
nom de machine \textbf{est identique à la casse près à}
la chaîne de caractères \verbtexte{S121-HP-P}.

\end{itemize}
%
Si jamais cela évoque quelque chose pour vous, sachez qu'en réalité le
nom des sous-répertoires est interprété par le client GNU/Linux comme une
\href{http://fr.wikipedia.org/wiki/Expression_rationnelle}{expression régulière étendue}.
Vous pouvez donc choisir comme nom de sous-répertoire n'importe quelle
expression régulière étendue pour filtrer les noms de machines qui
sont censées exécuter une fois vos scripts ou vos fichiers binaires.

Voici un dernier exemple de nom de sous-répertoire possible (et donc d'expression
régulière possible) :
\verbtexte{\string^.} (le nom de ce sous-répertoire est
constitué d'un accent circonflexe puis d'un point).
Cette expression régulière signifie : \og n'importe quelle
chaîne de caractères qui commence par un caractère quelconque \fg{}.
Autrement dit, les exécutables se trouvant dans ce sous-répertoire
seront lancés une fois au démarrage de \textbf{tous les clients GNU/Linux sans exception}.
Bien sûr, le répertoire \verbtexte{unefois/} du serveur peut parfaitement
contenir plusieurs sous-répertoires. Dans ce cas, si le nom de machine d'un client
correspond par exemple avec trois noms de sous-répertoires
\verbtexte{regex1/}, \verbtexte{regex2/} et \verbtexte{regex3/}, alors le client devra
lancer une seule fois au démarrage tous les exécutables contenus dans chacun des
sous-répertoires \verbtexte{regex1/}, \verbtexte{regex2/} et \verbtexte{regex3/}.

\begin{alerte}
Après avoir créé vos sous-répertoires et vos fichiers
exécutables dans le répertoire \verbtexte{unefois/} du serveur,
n'oubliez pas de réajuster les droits sur les fichiers comme expliqué
à la section~\ref{reconfigurer-droits} page~\pageref{reconfigurer-droits}.
\end{alerte}


Attention, les fichiers exécutables d'un sous-répertoire donné doivent
vérifier certains critères :
%
\begin{itemize}
\item Le nom d'un exécutable \textbf{ne doit pas commencer par un point}.
\item Le nom d'un exécutable \textbf{doit se terminer par \verbtexte{.unefois}}
(comme dans \verbtexte{mon-script.unefois}).
\item Si le fichier exécutable est un script (autrement dit si ce n'est pas un fichier
binaire), \textbf{il doit impérativement comporter un shebang}\footnote{Le shebang est la
première ligne d'un script qui commence par \verbtexte{\#!} comme
dans \og \verbtexte{\#! /bin/bash} \fg{} ou dans \og \verbtexte{\#! /usr/bin/python} \fg{}.} : 
cela peut-être un script Bash, Perl, Python peu importe
(du moment que l'interpréteur du langage est installé sur les
clients GNU/Linux)
mais il faut que le shebang soit présent.
\end{itemize}
%
\textbf{Le critère pour que les clients GNU/Linux se
souviennent d'avoir exécuté un fichier donné (afin de l'exécuter une seule fois) est le
nom de ce fichier}
et rien que le nom (pas le contenu). 
Par exemple, si un client GNU/Linux a exécuté le script \verbtexte{toto.unefois},
alors ce client n'exécutera plus jamais%
%
\footnote{En fait, comme vous allez le voir juste après,
cette règle n'est pas complètement immuable.}
%
de fichier s'appellant \verbtexte{toto.unefois}.
Si vous avez un script que vous souhaitez exécuter
non pas une seule fois, mais quelques fois de manière très ponctuelle
(une fois par an par exemple), pensez à
insérer la date du jour dans le nom du script (comme dans \verbtexte{1sept2012-maj.unefois})
et le cas échéant, en modifiant la date dans le nom du fichier
(par exemple en le renommant \verbtexte{3sept2013-maj.unefois}),
celui-ci sera à nouveau
candidat à l'exécution du côté des clients GNU/Linux.



\subsection{Le mécanisme en détail}

Voici le mécanisme effectué par les clients GNU/Linux
au niveau du répertoire \verbtexte{unefois/}
\textbf{au moment du démarrage du système} uniquement (le démarrage
est le seul instant où les clients GNU/Linux se préoccupent du répertoire
\verbtexte{unefois/}) :
%
\begin{enumerate}
\item Le client regarde le contenu de tous les sous-répertoires de
\verbtexte{/mnt/netlogon/unefois/}%
%
\footnote{Rappelons à nouveau
que le répertoire \verbtexte{/mnt/netlogon/unefois/} sur les clients GNU/Linux correspond
en réalité au répertoire \verbtexte{/home/netlogon/clients-linux/unefois/} du serveur Samba.}
%
dont les noms correspondent à son nom de machine. Par exemple, si le client
s'appelle \verbtexte{S18-DELL-03}, il va regarder le contenu du sous-répertoire
\verbtexte{\string^S18-} mais il va ignorer le sous-répertoire \verbtexte{-HP-}.
Dans chaque sous-répertoire qu'il n'a pas ignoré (s'il en existe),
le client va y chercher tous les fichiers de la forme \verbtexte{*.unefois},
afin d'obtenir toute une liste (éventuellement vide) de fichier \verbtexte{*.unefois}.


\item Si, dans cette liste de fichiers \verbtexte{*.unefois},
certains noms figurent déjà dans le répertoire local
\verbtexte{/etc/se3/unefois/}, c'est que les fichiers en question ont déjà
été exécutés par le client GNU/Linux et ils ne le sont donc pas une deuxième fois.
En revanche, les fichiers de cette liste dont le nom%
\footnote{Les clients GNU/Linux ne tiennent compte que du nom des
fichiers, pas de leur contenu. La casse dans le nom des fichiers est prise en compte.}
ne figure pas dans
\verbtexte{/etc/se3/unefois/} sont copiés dans ce répertoire local puis
les copies locales sont exécutées.
\end{enumerate}

C'est donc le répertoire \verbtexte{/etc/se3/unefois/}
qui constitue la \og mémoire \fg{}
du client GNU/Linux : il contient la liste des noms de fichiers déjà exécutés. Il y
a toutefois deux exceptions au mécanisme décrit ci-dessus :
%
\begin{enumerate}

\item Au moment du démarrage du système, si le client détecte la
présence d'un fichier nommé \verbtexte{PAUSE}%
%
\footnote{Le nom du fichier doit être
en majuscules uniquement et peu importe le contenu de ce fichier qui peut être
totalement vide. Attention, les droits de ce fichier doivent être corrects une fois
celui-ci créé.}
%
à la racine du répertoire \verbtexte{/mnt/netlogon/unefois/},
alors le client ne fait strictement
rien au niveau des fichiers \verbtexte{*.unefois}
et donc il n'exécute absolument rien, quoi qu'il arrive.

\item Au moment du démarrage du système, si le client
ne repère pas la présence du fichier \verbtexte{PAUSE} précédent mais qu'en
revanche il détecte la présence du fichier \verbtexte{BLACKOUT}%
%
\footnote{Même
remarque que pour le fichier \verbtexte{PAUSE}.},
%
toujours à la racine
du répertoire local \verbtexte{/mnt/netlogon/unefois/}, alors le client GNU/Linux efface
le contenu du répertoire \verbtexte{/etc/se3/unefois/}. Ainsi, au prochain
démarrage,
si les fichiers \verbtexte{PAUSE} et \verbtexte{BLACKOUT} ne sont pas présents,
le client exécutera tous les exécutables \verbtexte{*.unefois} qui le concerne,
peu importe leur nom
étant donné que la \og mémoire \fg{} du client GNU/Linux
concernant tout ce qui a déjà été exécuté a été effacée.
\end{enumerate}


\begin{alerte}
Au moment du démarrage, la recherche par les clients GNU/Linux
des fichiers \verbtexte{*.unefois} à exécuter (ainsi que leur 
copie en local le cas échéant) entraîne(nt) forcément du trafic réseau.
Lorsque vous ne souhaitez pas faire usage de ce mécanisme (ce qui en principe
sera le cas $90\%$ du temps),
n'hésitez pas à placer le fichier \verbtexte{PAUSE} à la racine
du répertoire \verbtexte{unefois/} du serveur afin d'éviter ce travail
de recherche aux clients GNU/Linux qui solliciteraient inutilement le
réseau.

Là encore, lorsque vous créerez ce fichier \verbtexte{PAUSE}, attention
de bien reconfigurer les droits des fichiers comme expliqué
à section~\ref{reconfigurer-droits} page~\pageref{reconfigurer-droits}.
\end{alerte}





Les scripts \verbtexte{*.unefois} sont
tous exécutés, en tant que \verbtexte{root}, \textbf{en arrière-plan}
et cela dès l'affichage de la fenêtre de connexion
lors du démarrage.
Si vous souhaitez qu'un script \verbtexte{*.unefois} se lance un peu après 
(parce que, par
exemple, vous avez besoin d'attendre que certains services soient
lancés), vous pouvez parfaitement utiliser des instructions comme
\og \verbtexte{sleep 20} \fg{} afin de forcer le script à attendre
pendant $20$ secondes avant de
commencer réellement son travail. 
Enfin sachez que dans le répertoire
local \verbtexte{/etc/se3/unefois/}, chaque exécutable \verbtexte{truc.unefois}
est accompagné de son homologue nommé \verbtexte{truc.unefois.log} qui contient
simplement l'ensemble des messages (d'erreur ou non) du fichier l'exécutable.




\subsection{Réglage de la locale durant l'exécution des scripts \og unefois \fg}

Avant de déployer un script bash via le répertoire \verbtexte{unefois/}
du serveur, il sera sans doute
nécessaire de le tester sur un client localement.
Sachez que les scripts bash, lorsqu'ils sont exécutés par le client
GNU/Linux au démarrage, ont la variable d'environnement \verbtexte{LC\_ALL}
définie comme étant égale à \verbtexte{C}%
\footnote{Cette valeur règle le système, le temps de l'exécution des scripts,
sur la locale standard \verbtexte{C} qui est la seule locale parfaitement
normalisée et a priori disponible sur n'importe quel système de type Unix.}
et non pas égale
à \verbtexte{fr\_FR.utf8}. Cela implique que tous les messages de sortie
des commandes système lancées dans le script
seront en anglais avec des caractères ASCII uniquement.
Pour avoir une idée de l'influence de la locale sur les commandes système,
vous pouvez ouvrir un terminal bash et tester ceci :
%
\begin{lstlisting}
# On paramètre le terminal sur une locale française qui doit être très
# probablement la locale par défaut déjà définie sur votre système.
export LC_ALL="fr_FR.utf8"
# Puis on teste une commande. En principe, l'entête du résultat de la
# commande est en français.
df -h

# Maintenant, on paramètre le terminal sur la locale C.
export LC_ALL="C"

# Et on teste à nouveau la même commande. Cette fois-ci, l'entête du
# résultat de la commande est en anglais.
df -h
\end{lstlisting}
%
Par conséquent, si jamais vous souhaitez exploiter le résultat de certaines
commandes système dans vos scripts bash \verbtexte{*.unefois}, sachez que la
locale peut avoir une incidence sur le comportement du script.
Si jamais vous
tenez à avoir une locale française lors de l'exécution de votre
script, alors il vous suffit de placer juste en dessous du shebang
l'instruction :
%
\begin{lstlisting}
export LC_ALL="fr_FR.utf8"
\end{lstlisting}
%
En revanche, si vous ne souhaitez pas forcer le réglage sur une
locale particulière et préférez conserver la valeur par défaut
(avec la locale standard \verbtexte{C}), alors durant vos tests afin
de valider un script bash à déployer, il faudra le lancer
de la manière suivante :
%
\begin{lstlisting}
LC_ALL="C" ./monscript.bash.unefois
\end{lstlisting}
%
De cette manière, le script héritera de la locale \verbtexte{C} et il
se comportera de la même manière que lors d'une exécution via le mécanisme
\og unefois \fg{}. Alors que si vous lancez le script ainsi :
%
\begin{lstlisting}
./monscript.bash.unefois
\end{lstlisting}
%
celui-ci hériterait de la locale du système, qui est très probablement
\verbtexte{fr\_FR.utf8}, et il se comporterait légèrement différemment que
lors d'une exécution via la mécanisme \og unefois \fg{}, si bien que vos 
tests seraient légèrement biaisés.


\subsection{Des variables et des fonctions prêtes à l'emploi}

Si jamais vous utilisez le langage Bash pour écrire des
script de la forme \verbtexte{*.unefois}, vous pouvez alors 
utiliser certaines variables ou fonctions prédéfinies
qui pourront peut-être vous faciliter le travail d'écriture
des scripts. Voici tableau listant toutes ces variables et
fonctions :
%
\begin{center}
\setlength{\LTcapwidth}{0.8\linewidth}
\renewcommand{\arraystretch}{1.5}
\begin{longtable}{|>{\small}c|>{\small}m{0.7\linewidth}|} \hline
\large \bfseries Nom & \centering \large \bfseries Commentaire \tabularnewline\hline
\endhead \hline
\caption{Variables et fonctions disponibles dans les scripts \og unefois \fg.}
\endfoot
\caption{Variables et fonctions disponibles dans les scripts \og unefois \fg.}
\label{tableau-unefois}
\endlastfoot
%
%
\verbtexte{SE3}
& 
Cette variable stocke l'adresse
IP du serveur récupérée automatiquement lors
de l'installation du paquet \verbtexte{se3-clients-linux}.
\\\hline
%
%
\verbtexte{NOM\_DE\_CODE} & 
Cette variable stocke ce qu'on appelle le
\og nom de code \fg{} de la distribution (\verbtexte{squeeze} dans
le cas d'une Debian Squeeze, \verbtexte{precise} dans le
cas d'une Ubuntu Precise Pangolin etc).
\\\hline
%
%
\verbtexte{ARCHITECTURE} &
Cette variable stocke l'architecture du système. Par exemple,
si le système repose sur une architecture $64$ bits, alors
la variable stockera la chaîne de caractères \verbtexte{x86\_64}.
\\\hline
%
%
\verbtexte{BASE\_DN} &
Cette variable contient le suffixe de base LDAP de l'annuaire
du serveur. Elle pourra vous être utile si vous souhaitez faire
vous-même des requêtes LDAP particulières sur les clients
à l'aide de la commande \verbtexte{ldapsearch}.
%
%
\\\hline
\verbtexte{NOM\_HOTE} &
Cette variable stocke le nom
du client GNU/Linux (celui qui se trouve dans le fichier
de configuration \verbtexte{/etc/hostname}).
Par exemple,
si vous avez pris l'habitude de choisir des noms de
machines de la forme \verbtexte{<salle>-xxx} (comme dans 
\verbtexte{S121-PC04} ou même comme dans \verbtexte{S18-DELL-02}), alors
vous pourrez récupérer le nom de la salle où se trouve le client
GNU/Linux par l'intermédiaire de la variable \verbtexte{NOM\_HOTE}
comme ceci :
%
\begin{lstlisting}
SALLE=$(echo "$NOM_HOTE" | cut -d'-' -f1)

if [ "$SALLE" = "S121" ]; then
   # Les trucs à faire si on est dans la salle 121.
fi

if [ "$SALLE" = "S18" ]; then
   # Les trucs à faire si on est dans la salle 18.
fi
# etc.
\end{lstlisting}
\\\hline
%
%
\verbtexte{appartient\_au\_parc} &
Cette fonction permet de savoir si une machine
appartient à un parc donné. Pour ce faire,
la fonction \verbtexte{appartient\_au\_parc} interroge l'annuaire
du serveur via une requête LDAP. Voici un exemple d'utilisation :
\begin{lstlisting}
if appartient_au_parc "S121" "$NOM_HOTE"; then
   # La machine appartient au parc S121
else
   # La machine n'appartient pas au parc S121
fi
\end{lstlisting}
% $
\\\hline
%
%
\verbtexte{afficher\_liste\_parcs} &
Un exemple vaudra mieux qu'un long discours :
\begin{lstlisting}
liste_parcs=$(afficher_liste_parcs "S121-LS-P")
\end{lstlisting}
% $
Dans cet exemple, la fonction effectue une requête LDAP
auprès du serveur afin de connaître le nom de tous les parcs
auxquels appartient la machine \verbtexte{S121-LS-P}.
Si la
machine \verbtexte{S121-LS-P} appartient aux parcs \verbtexte{S121}
et \verbtexte{PostesProfs}, alors la variable \verbtexte{liste\_parcs}
contiendra deux lignes, la première contenant \verbtexte{S121} et la
deuxième contenant \verbtexte{PostesProfs}. L'idée est de stocker tous
les parcs d'une machine dans une variable, le tout en une seule requête
LDAP. Enfin, à la place de \verbtexte{S121-LS-P} comme argument de la
fonction, on aurait pu utiliser \verbtexte{\$NOM\_HOTE}, comme dans
l'exemple ci-dessous qui sera plus éclairant sur la manière dont
on peut exploiter de telles listes.
\\\hline
%
%
\verbtexte{est\_dans\_liste} &
Là aussi, illustrons cette fonction par un exemple :
\begin{lstlisting}
# On récupère la liste des parcs auxquels 
# appartient la machine cliente.
liste_parcs=$(afficher_liste_parcs "$NOM_HOTE")

if est_dans_liste "$liste_parcs" "PostesProfs"; then
    # Si la machine est dans le parc "PostesProfs"
    # alors faire ceci...
elif est_dans_liste "$liste_parcs" "CDI"; then
    # Si la machine est dans le parc "CDI" 
    # alors faire ceci...
else
    # Sinon faire cela...
fi
\end{lstlisting}
% $
L'idée ici est qu'une seule requête LDAP est effectuée (lors de la
première instruction). Ensuite, les tests \verbtexte{if} ne sollicitent
pas le réseau puisque la liste des parcs est déjà stockée dans la
variable \verbtexte{liste\_parcs}.
\\\hline
%
%
\multicolumn{2}{|c|}{}\\
\multicolumn{2}{|c|}{%
\begin{minipage}{0.8\textwidth}
%\smaller[1]
Les fonctions suivantes sont moins pertinentes dans les scripts \verbtexte{*.unefois}
qui, rappelons-le, sont exécutés juste après le démarrage du système.
Mais elles restent toutefois disponibles également et donc figurent quand
même dans ce tableau. En revanche, nous verrons plus loin
(à la section~\ref{fonctions-utiles} page~\pageref{fonctions-utiles})
que ces fonctions
sont également disponibles à des moments beaucoup plus pertinents, comme
par exemple au moment de l'ouverture de session d'un utilisateur sur le système.
\end{minipage}
}\\
\multicolumn{2}{|c|}{}
\\\hline
%
%
\verbtexte{appartient\_au\_groupe} &
Cette fonction permet de savoir si le login d'un utilisateur
correspond à un compte qui appartient à un groupe donné. Pour ce faire,
la fonction \verbtexte{appartient\_au\_groupe} interroge l'annuaire
du serveur via une requête LDAP. Voici un exemple :
\begin{lstlisting}
if appartient_au_groupe "Classe_1ES2" "toto"; then
   # Le compte toto appartient à la classe 1ES2.
else
   # Le compte toto n'appartient pas à la classe 1ES2.
fi
\end{lstlisting}
% $
\\\hline
%
%
\verbtexte{afficher\_liste\_groupes} &
Un exemple vaudra mieux qu'un long discours :
\begin{lstlisting}
liste_groupes_toto=$(afficher_liste_groupes "toto")
if est_dans_liste "$liste_groupes_toto" "Eleves"; then
    # toto est un élève alors faire ceci...
fi
\end{lstlisting}
% $
Dans cet exemple, la fonction effectue une requête LDAP
auprès du serveur afin de connaître le nom des groupes auxquels
compte utilisateur \verbtexte{toto} appartient. Si par exemple ce
compte appartient aux groupes \verbtexte{Eleves}
et \verbtexte{Classe\_1ES2}, alors la variable \verbtexte{liste\_groupes\_toto}
contiendra deux lignes, la première contenant \verbtexte{Eleves} et la
deuxième contenant \verbtexte{Classe\_1ES2}. L'idée est de stocker tous
les groupes d'un compte donné dans une variable, le tout en une seule requête
LDAP. 
\\\hline
%
%
\verbtexte{est\_utilisateur\_local} &
Cette fonction permet de tester si un compte est local (c'est-à-dire
contenu dans le fichier \verbtexte{/etc/passwd} du client GNU/Linux) ou non (c'est-à-dire
un compte du domaine contenu dans l'annuaire du serveur).
\begin{lstlisting}
if est_utilisateur_local "toto"; then
    # toto est un compte local, alors faire ceci...
fi
\end{lstlisting}
\\\hline
%
%
\verbtexte{est\_connecte} &
Cette fonction permet de tester si un compte est actuellement
connecté au système (c'est-à-dire s'il a ouvert une session).
\begin{lstlisting}
if est_connecte "toto"; then
    # toto est actuellement connecté au système,
    # alors faire ceci...
fi
\end{lstlisting}
\\\hline
%
%
\verbtexte{activer\_pave\_numerique} &
Cette fonction, qui ne prend pas d'argument, permet simplement
d'activer le pavé numérique du client GNU/Linux.
\\\hline
%
%
\end{longtable}
\end{center}







\section{Le script de logon}
\label{logon-script}

\subsection{Phases d'exécution du script de logon}
\label{phase}

Le script de logon est un script bash qui est exécuté
par les clients GNU/Linux lors de trois phases
différentes. Pour plus de commodité dans les explications,
nous allons donner un nom à chacune de ces trois
phases une bonne fois pour toutes :
%
\begin{enumerate}
\item \textbf{L'initialisation :} cette phase se produit juste avant
l'affichage de la fenêtre de connexion. Attention, cela
correspond en particulier au démarrage du système, certes, mais pas seulement. 
L'initialisation se produit aussi
juste après la fermeture de session d'un utilisateur, avant que
la fenêtre de connexion n'apparaisse à nouveau  (sauf si, bien sûr,
l'utilisateur a choisi d'éteindre ou de redémarrer la machine).
\label{initialisation}
\smallskip

\textbf{Description rapide des tâches exécutées par le script lors de cette phase :}
le script efface les homes (s'il en existe)
de tous utilisateurs qui ne correspondent pas à des comptes locaux%
\footnote{Un compte local est un compte figurant dans le fichier
\verbtexte{/etc/passwd} du client GNU/Linux.}, vérifie si
le partage CIFS \verbtexte{//SERVEUR/netlgon-linux} du serveur
est bien monté sur le répertoire \verbtexte{/mnt/netlogon/} du client GNU/Linux et,
si ce n'est pas le cas, le script exécute ce montage. Ensuite,
le cas écheant, le script procède à la synchronisation
du profil par défaut local sur le profil par défaut distant et
lance les exécutions des \verbtexte{*.unefois} si l'initialisation
correspond en fait à un redémarrage du système.


\item \textbf{L'ouverture :} cette phase se produit à l'ouverture de 
session d'un utilisateur juste après que celui-ci ait saisi ses identifiants.
\label{ouverture}
\smallskip

\textbf{Description rapide des tâches exécutées par le script lors de cette phase :}
le script procède à la création
du home de l'utilisateur qui se connecte (via une
copie du profil par défaut local), exécute le montage
de certains partages du serveur auxquels l'utilisateur peut prétendre
(comme par exemple le partage correspondant aux données personnelles de
l'utilisateur).

\item \textbf{La fermeture :} cette phase se produit à 
la fermeture de session d'un utilisateur.
\label{fermeture}
\smallskip

\textbf{Description rapide des tâches exécutées par le script lors de cette phase :}
le script ne fait rien qui mérite d'être signalé dans cette documentation.
\end{enumerate}
%
Comme vous pouvez le constater, le script de logon est un peu le \og chef
d'orchestre \fg{} de chacun des clients GNU/Linux.

\subsection{Emplacement du script de logon}

À la base, le script de logon se trouve localement
à l'adresse \verbtexte{/etc/se3/bin/logon}
de chaque client GNU/Linux. Mais il existe une version centralisée
de ce script sur le serveur à l'adresse :
%
\begin{enumerate}
\item \verbtexte{\textcolor{red}{/home/netlogon/clients-linux/}bin/logon} si on est sur le serveur
\item \verbtexte{\textcolor{red}{/mnt/netlogon/}bin/logon} si on est sur un client GNU/Linux
\end{enumerate}
%
Nous avons donc, comme pour le profil par défaut, des versions
locales du script de logon (sur chaque client GNU/Linux)
et une unique version distante (sur le serveur).
Et au niveau de la synchronisation,
les choses fonctionnent de manière très similaire aux profils par défaut.
\textbf{Lors de l'initialisation d'un client GNU/Linux} :
%
\begin{itemize}
\item Si le contenu du script de logon local est identique au contenu
du script de logon distant, alors c'est le script de logon local qui est
exécuté par le client GNU/Linux.
\item Si en revanche les contenus diffèrent (ne serait-ce que d'un seul
caractère), alors c'est le script de logon distant qui est exécuté. Mais
dans la foulée, le script de logon local est écrasé puis remplacé par une
copie de la version
distante. Du coup, il est très probable qu'à la prochaine initialisation
du client GNU/Linux ce soit à nouveau le script de logon local qui soit exécuté
parce que identique à la version distante (on retombe dans le cas précédent).
\end{itemize}

A priori, cela signifie donc que, pour peu que vous sachiez parler (et écrire) le
langage du script de logon (il s'agit du Bash), vous pouvez modifier
\textbf{uniquement} le script de logon distant (celui du serveur donc)
afin de l'adapter à vos
besoins. Vos modifications seraient alors impactées sur \textbf{tous les clients}
GNU/Linux dès la prochaine phase d'initialisation.
Seulement, \textbf{il ne faudra pas procéder ainsi}
et cela pour une raison simple :
après la moindre mise à jour du paquet \verbtexte{se3-clients-linux} ou
éventuellement après une réinstallation, toutes vos modifications sur
le script de logon seront effacées. Pour pouvoir modifier le comportement du script
de logon de manière pérenne, il faudra utiliser le fichier
\verbtexte{logon\_perso} qui se trouve dans le même répertoire que le
script de logon.



\subsection{Personnaliser le script de logon}
\label{personnalisation}

Le fichier \verbtexte{logon\_perso} va vous permettre
d'affiner le comportement du script de logon afin de l'adapter
à vos besoins, et cela de manière pérenne dans le temps
(les modifications persisteront notamment après une mise
à jour du paquet \verbtexte{se3-clients-linux}). À la
base, le fichier \verbtexte{logon\_perso} est un fichier
texte encodé en UTF-8 avec des fins de ligne de type
Unix%
%
\footnote{Attention d'utiliser un éditeur de texte respectueux de l'encodage
et des fins de ligne lorsque vous modifierez le fichier \verbtexte{logon\_perso}.}.
%
Il contient du code bash et possède, par défaut, la structure suivante :
%
\begin{lstlisting}
function initialisation_perso ()
{
   # ...
}

function ouverture_perso ()
{
   # ...
}

function fermeture_perso ()
{
   # ...
}
\end{lstlisting}
%
Ce code sera ni plus ni moins inclus, tel quel, dans le script de logon.
En fait, après une modification du fichier \verbtexte{logon\_perso}, il
faudra donc \textbf{toujours} penser à reconfigurer le paquet
\verbtexte{se3-clients-linux} comme décrit dans la section~\ref{reconfigurer-droits}
page~\pageref{reconfigurer-droits} ce qui aura pour effet, entre autres,
d'insérer le contenu de la nouvelle version de \verbtexte{logon\_perso}
dans le fichier \verbtexte{logon}. Si vous oubliez de faire cette manipulation
après modification du fichier \verbtexte{logon\_perso}, le fichier \verbtexte{logon}
sera inchangé et vos modifications ne seront tout simplement pas prises en compte.

\begin{alerte}
\begin{center}
\textbf{Procédure à suivre quand on modifie le fichier \verbtexte{logon\_perso}}
\end{center}

Pour modifier le script de logon afin de l'adapter à vos besoins,
vous devez :
\begin{enumerate}
\item Modifier le fichier \verbtexte{logon\_perso}.
\item Puis lancer la reconfiguration du paquet \verbtexte{se3-clients-linux} en effectuant
une des deux procédures décrites dans la section~\ref{reconfigurer-droits}
page~\pageref{reconfigurer-droits}, afin que le contenu de la nouvelle version de
\verbtexte{logon\_perso} soit inséré dans le fichier \verbtexte{logon}.
\end{enumerate}
\end{alerte}


Revenons au contenu du fichier \verbtexte{logon\_perso} pour comprendre
de quelle manière il permet de modifier le comportement du script
\verbtexte{logon}. Dans le fichier \verbtexte{logon\_perso},
on peut distinguer trois fonctions :


\begin{enumerate}
\item Tout le code que vous mettrez dans la fonction \verbtexte{initialisation\_perso}
sera exécuté lors de la phase d'initialisation des clients, \textbf{en dernier}, c'est-à-dire
après que le script de logon ait effectué toutes les tâches
liées à la phase d'initialisation qui sont décrites
brièvement au point \ref{initialisation} de la section \ref{phase}.

\item Tout le code que vous mettrez dans la fonction \verbtexte{ouverture\_perso}
sera exécuté lors de la phase d'ouverture des clients uniquement lorsqu'un
utilisateur du domaine se connecte. Le code est exécuté \textbf{juste après}
la création du \og home \fg{} de l'utilisateur qui se connecte. 
Typiquement, c'est dans cette fonction que vous allez gérer les montages
de partages réseau en fonction du type de compte qui se connecte (son
appartenance à tel ou tel groupe etc).

Pour la gestion des montages de partages réseau à l'ouverture de session,
tout se trouve à la section~\ref{gestion-montage} page~\pageref{gestion-montage}.


\item Tout le code que vous mettrez dans la fonction \verbtexte{fermeture\_perso}
sera exécuté lors de la phase de fermeture des clients, \textbf{en dernier}, c'est-à-dire
après que le script de logon ait effectué toutes les tâches
liées à la phase de fermeture qui sont décrites
brièvement au point \ref{fermeture} de la section \ref{phase}.

\end{enumerate}

Vous pouvez bien sûr définir dans le fichier \verbtexte{logon\_perso}
des fonctions supplémentaires, mais, pour que celles-ci soient
au bout du compte exécutées
par le script de logon, il
faudra les appeler dans le corps d'une des trois
fonctions \verbtexte{initialisation\_perso}, \verbtexte{ouverture\_perso}
ou \verbtexte{fermeture\_perso}.

Il faut bien avoir en tête que le contenu de \verbtexte{logon\_perso}
est ni plus ni moins inséré dans le script \verbtexte{logon} et donc, après
modification de \verbtexte{logon\_perso}, il faut toujours mettre à jour
le fichier \verbtexte{logon} via la commande 
\og \verbtexte{dpkg-reconfigure se3-clients-linux} \fg{}.



\subsection{Quelques variables et fonctions prêtes à l'emploi}
\label{fonctions-utiles}

Voici la liste des variables et des fonctions que vous
pourrez utiliser dans le fichier \verbtexte{logon\_perso} et qui
seront susceptibles de vous aider à affiner le comportement
du script de logon :
%
\begin{center}
\setlength{\LTcapwidth}{0.8\linewidth}
\renewcommand{\arraystretch}{1.5}
\begin{longtable}{|>{\small}c|>{\small}m{0.7\linewidth}|} \hline
\large \bfseries Nom & \centering \large \bfseries Commentaire \tabularnewline\hline
\endhead \hline
\caption{Variables et fonctions disponibles dans le fichier logon\_perso.}
\endfoot
\caption{Variables et fonctions disponibles dans le fichier logon\_perso.}
\endlastfoot
%
%
\multicolumn{2}{|c|}{}\\
\multicolumn{2}{|c|}{%
\begin{minipage}{0.8\textwidth}
Pour commencer, toutes les variables et les fonctions présentées dans le
tableau~\ref{tableau-unefois} à la page~\pageref{tableau-unefois} sont
utilisables.
\end{minipage}
}\\
\multicolumn{2}{|c|}{}\\\hline
%
%
\verbtexte{LOGIN} &
Cette variable stocke le login
de l'utilisateur qui a ouvert une session. Cette
variable n'a de sens que lors de la phase d'ouverture
et de fermeture (c'est-à-dire uniquement à l'intérieur
des fonctions \verbtexte{ouverture\_perso} et \verbtexte{fermeture\_perso}),
pas lors de la phase d'initialisation (c'est-à-dire à l'intérieur de la fonction
\verbtexte{initialisation\_perso}) puisque 
personne n'a encore ouvert de session à ce moment là.
\\\hline
%
%
\verbtexte{NOM\_COMPLET\_LOGIN} &
Cette variable stocke le nom complet (sous la forme \og prénom nom \fg{})
de l'utilisateur qui a ouvert une session. Cette
variable n'a de sens que lors de la phase d'ouverture et de fermeture.
\\\hline
%
%
\verbtexte{REP\_HOME} &
Cette variable stocke le chemin absolu du répertoire home 
de l'utilisateur qui se connecte. Par exemple, si le compte
\verbtexte{toto} ouvre une session, la variable contiendra
la chaîne \verbtexte{/home/toto}. Remarquez que cette variable
est un simple raccourci pour écrire \verbtexte{"/home/\$LOGIN"}.
Cette variable n'a de sens que lors de la phase d'ouverture et de fermeture.
\\\hline
%
%
\verbtexte{LISTE\_GROUPES\_LOGIN} &
Cette variable, qui n'a de sens \textbf{que lors de la phase d'ouverture},
stocke la liste des groupes auxquels appartient l'utilisateur qui a
ouvert une session (le format étant un nom de groupe par ligne). Une
utilisation typique de cette variable est :
\begin{lstlisting}
if est_dans_liste "$LISTE_GROUPES_LOGIN" "Profs"; then
    # L'utilisateur qui se connecte appartient au
    # groupe Profs, alors faire ceci...
elif est_dans_liste "$LISTE_GROUPES_LOGIN" "Eleves"; then
    # L'utilisateur qui se connecte appartient au
    # groupe Eleves, alors faire cela...
fi
\end{lstlisting} %$
Au passage, dans ce code, aucune requête LDAP n'est effectuée
puisque la variable \verbtexte{LISTE\_GROUPES\_LOGIN} contient
déjà la liste des groupes auxquels appartient l'utilisateur qui
vient de se connecter (la requête LDAP permettant de définir
la variable \verbtexte{LISTE\_GROUPES\_LOGIN} a été faite 
par le script de logon en amont, une fois pour toute).
\\\hline
%
%
\verbtexte{DEMARRAGE} &
Cette variable stocke toujours la valeur \verbtexte{false},
sauf lorsqu'on se trouve lors d'une phase d'initialisation
qui correspond à un démarrage du système où elle stocke alors
la valeur \verbtexte{true}. Cette variable n'a
donc d'intérêt que lorsqu'elle est utilisée dans la fonction
\verbtexte{initialisation\_perso}. Voici un exemple :
\begin{lstlisting}
if "$DEMARRAGE"; then
    # On est lors d'une phase de démarrage
    # alors faire ceci...
fi
\end{lstlisting} %$
\\\hline
%
%
\verbtexte{monter\_partage} &
Si vous voulez que les utilisateurs du domaine puissent avoir
accès à des partages réseau sur le serveur, il faudra forcément
faire usage de cette fonction qui est donc très importante.
Toutes les explications sur cette fonction se trouvent
à la section~\ref{gestion-montage} page~\pageref{gestion-montage}.
Cette fonction n'a de sens que lors de la phase d'ouverture.
\\\hline
%
%
\verbtexte{creer\_lien} &
Cette fonction, qui va de pair avec la précédente, sera
détaillée à la section~\ref{gestion-montage}
page~\pageref{gestion-montage}.
Cette fonction n'a de sens que lors de la phase d'ouverture.
\\\hline
%
%
\verbtexte{changer\_icone} &
Cette fonction sera
détaillée à la section~\ref{changer-icone}
page~\pageref{changer-icone}.
Cette fonction n'a de sens que lors de la phase d'ouverture.
\\\hline
%
%
\verbtexte{changer\_papier\_peint} &
Cette fonction, utilisable uniquement pendant la phase d'ouverture,
permet de changer le fond d'écran de l'utilisateur qui se connecte.
Elle prend un argument qui correspond au chemin absolu (sur le
client) du fichier image à utiliser en guise de fond d'écran.
Un exemple de l'utilisation de cette fonction sera
donné à la section~\ref{papier-peint}
page~\pageref{papier-peint}.
\\\hline
%
%
\verbtexte{activer\_pave\_numerique} &
Cette fonction, qui fait exactement ce à quoi on pense naturellement, sera
détaillée à la section~\ref{pave-num} page~\pageref{pave-num}.
\\\hline
%
%
\verbtexte{executer\_a\_la\_fin} &
Parfois certaines commandes nécessitent d'être exécutées un fois le script
de logon terminé (c'est-à-dire une fois l'initialisation, l'ouverture
ou la fermeture terminée). C'est ce que permet cette fonction. Avec
par exemple :
\begin{lstlisting}
executer_a_la_fin "5" "commande" "arg1" "arg2"
\end{lstlisting}
la commande \verbtexte{commande} (avec ses arguments)
sera lancée 5 secondes après que
le script de logon ait terminé son exécution.
Un exemple de l'usage de cette fonction sera
donné à la section~\ref{conky} page~\pageref{conky}.
Attention, l'exécution se faisant une fois le script de logon terminé,
il y aura aucune trace dans les fichiers de log de l'exécution
de la commande \verbtexte{commande}.
\\\hline
%
%
\end{longtable}
\end{center}




\subsection{Gestion du montage des partages réseau}
\label{gestion-montage}

Comme cela a déjà été expliqué, c'est vous qui allez gérer
les montages de partages réseau en éditant le contenu de la
fonction \verbtexte{ouverture\_perso} qui se trouve dans le fichier
\verbtexte{logon\_perso}. Évidemment, si la gestion par défaut
des montages vous convient telle quelle, alors vous n'avez pas besoin
de toucher à ce fichier.
Commençons par un exemple simple :
%
\begin{lstlisting}
function ouverture_perso ()
{
    monter_partage "//$SE3/Classes" "Classes" "$REP_HOME/Bureau/Répertoire Classes"
}
\end{lstlisting}
%
Ici la fonction \verbtexte{monter\_partage}
possède trois \textbf{arguments qui devront être délimités
par des doubles quotes} (\verbtexte{"}) :
%
\begin{enumerate}
\item Le premier représente le chemin UNC du partage à monter.
Vous reconnaissez sans doute la variable \verbtexte{SE3} qui
stocke l'adresse IP du serveur. Par exemple si l'adresse IP
du serveur est \verbtexte{172.20.0.2}, alors le premier argument 
sera automatiquement développé en :
%
\begin{center}
\verbtexte{//172.20.0.2/Classes}.
\end{center}
%
Cela signifie que c'est le partage \verbtexte{Classes} du serveur
\verbtexte{172.20.0.2} qui va être monté sur le clients GNU/Linux.
Attention, sous GNU/Linux un chemin UNC de partage s'écrit avec des
slashs (\verbtexte{/}) et non avec des antislashs 
(\verbtexte{$\backslash$}) comme c'est le cas
sous Windows.

\item Maintenant, il faut un répertoire local pour monter un partage.
C'est le rôle du deuxième argument. Quoi qu'il arrive (vous n'avez pas
le choix sur ce point), le partage
sera monté dans un sous-répertoire du répertoire \verbtexte{/mnt/\_\$LOGIN/}.
Par exemple si c'est \verbtexte{toto} qui se connecte sur le poste
client, le montage sera fait dans un sous répertoire de \verbtexte{/mnt/\_toto/}.
Le deuxième argument spécifie le nom de ce sous-répertoire. Ici nous
avons décidé assez logiquement de l'appeler \verbtexte{Classes}.
Par conséquent, en visitant le répertoire \verbtexte{/mnt/\_toto/Classes/}
sur le poste client, notre cher \verbtexte{toto} aura accès au
contenu du partage \verbtexte{Classes} du serveur.

Attention, dans le choix du nom de ce sous-répertoire, vous êtes
limité(e) aux \textbf{caractères a-z, A-Z, 0-9, le tiret (\verbtexte{-})
et le tiret bas (\verbtexte{\_})}.
C'est tout. En particulier \textbf{pas d'espace ni accent}.
Si vous ne respectez pas cette consigne le partage ne sera tout
simplement pas monté et une fenêtre d'erreur s'affichera à l'ouverture
de session.

Vous serez sans doute amené(e) à monter plusieurs partages réseau
pour un même utilisateur (via plusieurs appels de la fonction
\verbtexte{monter\_partage} au sein de la fonction 
\verbtexte{ouverture\_perso}). Donc il y aura plusieurs sous-répertoires
dans \verbtexte{/mnt/\_\$LOGIN/}. Charge à vous d'éviter les doublons
dans les noms des sous-répertoires, sans quoi certains partages ne
seront pas montés.


\item À ce stade, notre cher \verbtexte{toto} pourra accéder au
partage \verbtexte{Classes} du serveur en passant par
\verbtexte{/mnt/\_toto/Classes/}. Mais cela n'est pas très pratique.
L'idéal serait d'avoir accès à ce partage directement via un dossier
sur le bureau de \verbtexte{toto}. C'est exactement ce que fait
le troisième argument. Si \verbtexte{toto} ouvre
une session, l'argument \verbtexte{"\$REP\_HOME/Bureau/Répertoire Classes"}
va se développer en \verbtexte{"/home/toto/Bureau/Répertoire Classes"}
si bien qu'un raccourci (sous GNU/Linux on appelle ça un lien symbolique)
portant le nom \verbtexte{Répertoire Classes}
sera créé sur le bureau de \verbtexte{toto}. Donc en double-cliquant sur
ce raccourci (vous pouvez voir à la page~\pageref{captureic} 
via une capture d'écran que ce
genre de raccourci ressemble à un simple dossier), sans même
le savoir, \verbtexte{toto} visitera le répertoire
\verbtexte{/mnt/\_toto/Classes/} qui correspondra au contenu du 
partage \verbtexte{Classes} du serveur. Vous n'êtes pas limité(e)
dans le choix du nom de ce raccourci. Les espaces et les accents
sont parfaitement autorisés (évitez par contre le caractère double-quote).
En revanche, ce raccourci doit forcément être créé dans le home
de l'utilisateur qui se connecte. \textbf{Donc ce troisième argument devra
toujours commencer par \verbtexte{"\$REP\_HOME/..."}} sans quoi le lien
ne sera tout simplement pas créé.
\end{enumerate}


Tout n'a pas encore été dévoilé concernant cette fonction 
\verbtexte{monter\_partage}. En fait, vous pouvez créer autant
de raccourcis que vous voulez. Il suffit pour cela d'ajouter un
quatrième argument, puis un cinquième , puis un sixième etc.
Voici un exemple :
%
\begin{lstlisting}[emph={ENTREE},emphstyle={\return}]
function ouverture_perso ()
{
    monter_partage "//$SE3/Classes" "Classes" \ENTREE
        "$REP_HOME/Bureau/Lecteur réseau Classes" \ENTREE
        "$REP_HOME/Lecteur réseau Classes"
}
\end{lstlisting}
%$
%
\begin{RQ}
normalement il faut mettre une fonction avec ses arguments sur une
même ligne car un saut de ligne signifie la fin d'une instruction
aux yeux de l'interpréteur Bash. Mais ici la ligne serait bien longue à écrire
et dépasserait la largeur de la page de ce document.
La combinaison antislash (\verbtexte{$\backslash$}) puis ENTRÉE permet simplement
de passer à la ligne tout en signifiant à l'interpréteur Bash que l'instruction
entamée n'est pas terminée et qu'elle se prolonge sur la ligne suivante.
\end{RQ}
%
Le premier argument correspond toujours au chemin UNC du partage
réseau et le deuxième argument au nom du sous-répertoire 
dans \verbtexte{/mnt/\_\$LOGIN/} associé à ce partage. Ensuite, nous
avons cette fois-ci un troisième \textbf{et un quatrième argument} qui
correspondent aux raccourcis pointant vers le partage : l'un
est créé sur le bureau et l'autre est créé à
la racine du home de l'utilisateur qui se connecte.
Il est possible de créer autant de raccourcis que l'on souhaite,
il suffit d'empiler les arguments $3$, $4$, $5$ etc. 
les uns à la suite des autres.

La syntaxe de la fonction \verbtexte{monter\_partage} est donc
la suivante :
%
\begin{lstlisting}
monter_partage "<partage>" "<répertoire>" ["<raccourci>"]...
\end{lstlisting}
%
où seuls les deux premiers arguments sont obligatoires :
\begin{itemize}
\item \verbtexte{<partage>} est le chemin UNC du partage à monter.
Il est possible de se limiter à un sous-répertoire du partage,
par exemple comme dans \verbtexte{//\$SE3//administration/docs} où l'on
montera uniquement le sous-répertoire \verbtexte{docs/} du
partage \verbtexte{administration} du serveur.

\item \verbtexte{<répertoire>} est le nom du sous-répertoire
de \verbtexte{/mnt/\_\$LOGIN/} qui sera créé et sur lequel le
partage sera monté. Seuls les caractères \verbtexte{-\_a-zA-Z0-9}
sont autorisés.

\item Les arguments \verbtexte{<raccourci>} sont optionnels. Ils
représentent les chemins absolus des raccourcis qui seront créés
et qui pointeront vers le partage. Ils doivent toujours se situer
dans le home de l'utilisateur qui se connecte, donc ils doivent
toujours commencer par \verbtexte{"\$REP\_HOME/..."}. Si ces arguments
ne sont pas présents, alors le partage sera monté mais 
aucun raccourci ne sera créé.
\end{itemize}

\begin{alerte}
Attention, le montage du partage réseau se fait avec les droits
de l'utilisateur qui est en train de se connecter. Si l'utilisateur
n'a pas les droits suffisants pour accéder à ce partage, ce dernier
ne sera tout simplement pas monté.
\end{alerte}

\begin{RQ}
au final, si vous placez bien vos raccourcis, l'utilisateur
n'aura que faire du répertoire \verbtexte{"/mnt/\_\$LOGIN/"}. Il
utilisera uniquement les raccourcis qui se trouvent dans son home. Peu importe
pour lui de savoir qu'ils pointent en réalité vers un sous-répertoire
de \verbtexte{"/mnt/\_\$LOGIN/"}, il n'a pas à s'en préoccuper.
\end{RQ}

\begin{RQ}
je vous conseille de toujours créer au moins un raccourci à
la racine du home de l'utilisateur qui se connecte. En effet,
lorsqu'un utilisateur souhaite enregistrer un fichier via une application quelconque,
très souvent l'explorateur de fichiers s'ouvre au départ à la racine de son
home. C'est donc un
endroit privilégié pour placer les raccourcis vers les partages réseau.
Il me semble que doubler les raccourcis à la fois à la racine du home
et sur le bureau de l'utilisateur est une bonne chose.
Mais bien sûr, tout cela est une question de goût...
\end{RQ}


Étant donné que le montage d'un partage se fait
avec les droits de l'utilisateur qui se connecte, certains partages
devront être montés uniquement dans certains cas.
Prenons l'exemple du partage \verbtexte{netlogon-linux}
du serveur. Celui-ci n'est accessible qu'au compte \verbtexte{admin} du
domaine. Pour pouvoir monter ce partage seulement quand c'est
le compte \verbtexte{admin} qui se connecte, il va falloir
ajouter ce bout de code dans la fonction
\verbtexte{ouverture\_perso} du fichier \verbtexte{logon\_perso} :
%
\begin{lstlisting}[emph={ENTREE},emphstyle={\return}]
function ouverture_perso ()
{
    # Montage du partage "netlogon-linux" seulement dans le cas 
    # où c'est le compte "admin" qui se connecte.
    if [ "$LOGIN" = "admin" ]; then
        # Cette partie là ne sera exécutée qui si c'est admin qui se connecte.
        monter_partage "//$SE3/netlogon-linux" "clients-linux" \ENTREE
            "$REP_HOME/clients-linux" \ENTREE
            "$REP_HOME/Bureau/clients-linux"
    fi
}
\end{lstlisting}
%$
%
\begin{RQ}
attention, en Bash, le crochet ouvrant au niveau du \verbtexte{if} doit
absolument être précédé et suivi d'un espace et le crochet fermant doit
absolument être précédé d'un espace.
\end{RQ}
%
Autre cas très classique, celui d'un partage \textbf{accessible uniquement à
un groupe}. Là aussi, une structure avec un \verbtexte{if} s'impose :
%
\begin{lstlisting}[emph={ENTREE},emphstyle={\return}]
function ouverture_perso ()
{
    # On décide que le montage du partage "administration" sera seulement effectué si
    # c'est un compte qui appartient au groupe "Profs" qui se connecte.
    if est_dans_liste "$LISTE_GROUPES_LOGIN" "Profs"; then
        monter_partage "//$SE3/administration" "administration" \ENTREE
            "$REP_HOME/administration sur le réseau" \ENTREE
            "$REP_HOME/Bureau/administration sur le réseau"
    fi
}
\end{lstlisting}
%
L'instruction \og \verbtexte{if est\_dans\_liste "\$LISTE\_GROUPES\_LOGIN" "Profs"; then} \fg{}
doit s'interpréter ainsi : \og si dans la liste des groupes dont est membre le compte qui
se connecte actuellement il y a le groupe \verbtexte{Profs}, autrement dit si
le compte qui se connecte actuellement appartient au groupe \verbtexte{Profs},
alors... \fg{}

\begin{alerte}
Attention, le test \verbtexte{if} ci-dessus est sensible à la casse
si bien que le résultat ne sera pas le même si vous mettez \verbtexte{"Profs"}
ou \verbtexte{"profs"}. Par conséquent, prenez bien la peine de regarder
le nom du groupe qui vous intéresse avant de l'insérer dans un test \verbtexte{if}
comme ci-dessus afin de bien respecter les minuscules et les majuscules.
\end{alerte}

Si vous voulez savoir le nom des partages disponibles pour
un utilisateur donné, par exemple \verbtexte{toto}, il vous
suffit de lancer la commande suivante sur le serveur en tant
que \verbtexte{root} :
%
%
\begin{lstlisting}
smbclient --list localhost -U toto
# Il faudra alors saisir le mot de passe de toto.
\end{lstlisting}
%
Parmi la liste des partages, l'un d'eux est affiché sous le nom
de \verbtexte{home}. Il correspond au home de \verbtexte{toto}
sur le serveur. Ce partage est un peu particulier car il pointera
vers un répertoire différent en fonction du compte qui tente d'y
accéder. Par exemple, si \verbtexte{titi} veut accéder à ce partage,
alors il sera rédirigé vers le répertoire \verbtexte{/home/titi/} du serveur.
Chaque utilisateur a le droit de monter ce partage,
mais attention le chemin UNC est en fait
\verbtexte{//SERVEUR/homes} (avec un \og s \fg{} à la fin et d'ailleurs
dans le fichier de configuration Samba ce partage est bien défini
par la section \verbtexte{homes}).
A priori, on pourra monter ce partage pour tous les comptes
du domaine donc pas besoin de structure \verbtexte{if} pour
ce partage :
%
\begin{lstlisting}[emph={ENTREE},emphstyle={\return}]
function ouverture_perso ()
{
    # Montage du sous-répertoire "Docs" du partage "homes" pour tout le monde.
    monter_partage "//$SE3/homes/Docs" "Docs" \ENTREE
        "$REP_HOME/Documents de $LOGIN sur le réseau" \ENTREE
        "$REP_HOME/Bureau/Documents de $LOGIN sur le réseau"
}
\end{lstlisting}
%$
%
Dans l'exemple ci-dessus, on ne monte pas le partage \verbtexte{homes}
mais uniquement le sous-répertoire \verbtexte{Docs} de ce partage.
Comme d'habitude sous GNU/Linux, respectez bien la casse des noms de
partages et de répertoires.

Pour l'instant, de par la manière dont la fonction
\verbtexte{monter\_partage} est définie, on peut créer uniquement
des liens qui pointent vers la racine du partage associé. Mais on
peut vouloir  par exemple monter un partage et créer des liens uniquement
vers des sous-répertoires de ce partage (et non vers sa racine). C'est tout
à fait possible avec la fonction \verbtexte{creer\_lien}. Voici un exemple :
%
\begin{lstlisting}[emph={ENTREE},emphstyle={\return}]
function ouverture_perso ()
{
    # Montage du partage "homes" pour tout le monde, mais ici on ne créé pas de
    # lien vers la racine de ce partage (appel de la fonction avec seulement deux
    # arguments).
    monter_partage "//$SE3/homes" "home"
    
    # Ensuite on crée des liens mais ceux-ci ne pointent pas à la racine du partage.
    creer_lien "home/Docs" "$REP_HOME/Documents de $LOGIN sur le réseau"
    creer_lien "home/Bureau" "$REP_HOME/Bureau de $LOGIN sous Windows"
}
\end{lstlisting}
%$
%
Le premier argument de la fonction \verbtexte{creer\_lien} est la cible
du ou des liens à créer. Cette cible peut s'écrire sous la forme
d'un chemin absolu, c'est-a-dire un chemin qui commence par un antislash (ce qui
n'est pas le cas ci-dessus). Si le chemin ne commence pas par un antislash,
alors la fonction part du principe que c'est un chemin relatif qui part
de \verbtexte{/mnt/\_\$LOGIN/}%
%
\footnote{Du coup, mettre \verbtexte{"home/Docs"} ou mettre
\verbtexte{/mnt/\_\$LOGIN/home/Docs} comme premier argument
revient exactement au même.}.
%
Ensuite, le deuxième argument et les
suivants (autant qu'on veut) sont les chemins absolus du ou des
liens qui seront
créés. Ces chemins doivent impérativement tous commencer par
\verbtexte{"\$REP\_HOME/..."}.







\subsection{Quelques bricoles pour les perfectionnistes}







\subsubsection{Changer les icônes représentants les liens pour faire plus joli}
\label{changer-icone}

C'est quand même plus joli quand on a des icônes évocateurs%
%
\footnote{En informatique, le masculin est autorisé pour le mot icône.}
%
comme ci-dessous pour nos liens vers les partages, non ?
%
\begin{center}
\includegraphics[width=0.2\linewidth]{icones_jolis1}
\label{captureic}
\end{center}
%
Et bien ça tombe bien car 
c'est facile à faire avec la fonction \verbtexte{changer\_icone}.
Voici un exemple :
%
\begin{lstlisting}[emph={ENTREE},emphstyle={\return}]
function ouverture_perso ()
{
    # On suppose que le partage "Classe" est déjà monté et qu'un
    # lien vers ce partage a déjà été créé sur le bureau...
    changer_icone "$REP_HOME/Bureau/Classes sur le réseau" \ENTREE
                  "$REP_HOME/.mes_icones/classe.jpg"
}
\end{lstlisting}
%
La fonction prend toujours deux arguments. Le premier est le chemin absolu
du fichier dont on veut changer l'icône. Cela peut être n'importe
quel fichier (ce n'est pas forcément un des raccourcis qu'on a créé),
mais par contre il doit impérativement se trouver dans le home
de l'utilisateur qui se connecte (donc il devra toujours commencer
par \verbtexte{"\$REP\_HOME/..."}). Ensuite, le deuxième argument
est le chemin absolu de n'importe quel fichier image (du moment
que le compte qui se connecte peut y avoir accès en lecture).

Une idée possible (parmi d'autres) est de modifier le profil par
défaut des d'utilisateurs et d'y placer un répertoire 
\verbtexte{.mes\_icones/} dans lequel vous mettez tous les icônes
dont vous avez besoin pour habiller vos liens. Ensuite, vous
pourrez aller chercher vos icônes dans le home de l'utilisateur
qui se connecte (dans \verbtexte{"\$REP\_HOME/.mes\_icones/"} précisément)
de manière similaire à ce qui est fait dans exemple
ci-dessus.

\begin{alerte}
Attention, la fonction \verbtexte{changer\_icone} n'a aucun effet
sous la distribution Xubuntu qui utilise l'environnement de bureau
Xfce. Cela vient du fait que personnellement je ne sais pas
changer l'image d'un icône en ligne de commandes sous Xfce. Si vous
savez, n'hésitez pas à me donner l'information par mail
car je pourrais ainsi étendre la fonction \verbtexte{changer\_icone}
à l'environnement de bureau Xfce.
\end{alerte}




\subsubsection{Changer le papier peint en fonction des utilisateurs}
\label{papier-peint}

Ça pourrait être sympathique d'avoir un papier différent
suivant le type de compte... Et bien c'est possible avec la
fonction \verbtexte{changer\_papier\_peint}. Voici un exemple :
%
\begin{lstlisting}[emph={ENTREE},emphstyle={\return}]
function ouverture_perso ()
{
    if [ "$LOGIN" = "admin" ]; then
        changer_papier_peint "$REP_HOME/.backgrounds/admin.jpg"
    fi
}
\end{lstlisting} %$
%
Le seul et unique argument de cette fonction est le chemin absolu
(sur la machine cliente) du fichier image servant pour le fond d'écran.
Il faut bien sûr que ce fichier image soit au moins accessible en
lecture pour l'utilisateur qui se connecte.

Là aussi, comme pour les icônes, l'idée est de placer dans le
profil par défaut distant un répertoire \verbtexte{.backgrounds/}
(par exemple) qui contiendra les deux ou trois fichiers images
dont vous avez besoin pour faire vos fonds d'écran. Voici un
exemple dans le cas d'un compte professeur :
%
\begin{center}
\includegraphics[width=0.8\linewidth]{bureau-message}
\end{center}
%
En plus du changement de fond d'écran, il y a un petit message
personnalisé qui s'affiche en haut à droite du bureau.
Pour mettre en place ce genre de message, voir la
section~\ref{conky} page~\pageref{conky}.

\subsubsection{L'activation du pavé numérique}
\label{pave-num}

Pour activer le pavé numérique du client GNU/Linux
au moment de l'affichage de la
fenêtre de connexion du système, en principe ceci devrait
fonctionner :
%
\begin{lstlisting}
function initialisation_perso ()
{
    # On active le pavé numérique au moment de la phase d'initialisation.
    activer_pave_numerique
}
\end{lstlisting}
%
Vous pouvez remarquer que, cette fois-ci, c'est le contenu
de la fonction \verbtexte{initialisation\_perso} qui a été
édité.

En revanche, pour activer le pavé numérique au moment de
l'ouverture de session, procéder exactement de la même façon à l'intérieur
de la fonction \verbtexte{ouverture\_perso} risque de ne pas
fonctionner, et cela pour une raison de timing.
En effet, au moment où la fonction \verbtexte{ouverture\_perso}
sera lancée, l'ouverture de session ne sera pas complètement
terminée%
%
\footnote{Et c'est normal qu'il en soit ainsi puisque l'ouverture
de session de termine \textbf{après} l'exécution du script de logon, même
pas immédiatement après mais 1 ou 2 secondes après selon la
rapidité de la machine hôte.}
%
et l'activation du pavé numérique risque d'être
annulée lors de la fin de l'ouverture de session. L'idée est
donc de programmer l'appel de la fonction \verbtexte{activer\_pave\_numerique}
\textbf{après} l'exécution du script de logon,
seulement au bout de quelques secondes (par exemple $5$), afin de
lancer l'activation du pavé numérique une fois l'ouverture de session
achevée :
%
\begin{lstlisting}
function ouverture_perso ()
{
    # On ajoute un argument à l'appel de la fonction activer_pave_numerique.
    # Ici, cela signifie que l'activation du pavé numérique sera lancée 5
    # secondes après que le script de logon soit terminé, ce qui laissera
    # le temps à l'ouverture de session de se terminer.
    activer_pave_numerique "5"
}
\end{lstlisting}
%


\subsubsection{Incruster un message sur le bureau des utilisateurs pour faire classe}
\label{conky}

Pour incruster un message sur le bureau des utilisateurs, il
faudra d'abord que le paquet \verbtexte{conky} soit installé%
\footnote{Vous pouvez par exemple lancer l'installation via
un script \verbtexte{*.unefois} qui contiendrait à peu de
choses près l'instruction \verbtexte{apt-get install {-}{-}yes conky}.}
sur le client GNU/Linux.
Ensuite, tentez de mettre ceci dans
la fonction \verbtexte{ouverture\_perso} :
%
\begin{lstlisting}
function ouverture_perso ()
{
    # On crée un fichier de configuration .conkyrc dans le home de l'utilisateur.
    # précisant le contenu du message ainsi que certains paramètres (comme la
    # taille de la police par exemple).
    cat > "$REP_HOME/.conkyrc" <<FIN
use_xft yes
xftfont Arial:size=10
double_buffer yes
alignment top_right
update_interval 1
own_window yes
own_window_transparent yes
override_utf8_locale yes
text_buffer_size 1024
own_window_hints undecorated,below,sticky,skip_taskbar,skip_pager
TEXT
Bonjour $NOM_COMPLET_LOGIN,
Pensez bien à enregistrer vos données personnelles
dans le dossier :

     Documents de $LOGIN sur le réseau
     
qui se trouve sur le bureau, et uniquement dans ce
dossier, sans quoi vos données seront perdues une
fois votre session fermée.

   Cordialement.
   Les administrateurs du réseau pédagogique.
FIN

    # On fait de "$LOGIN" le propriétaire du fichier .conkyrc.
    chown "$LOGIN:" "$REP_HOME/.conkyrc"
    chmod 644 "$REP_HOME/.conkyrc"
    
    # On lancera conky à la fin, une fois l'exécution du script logon terminée.
    # Pour être sûr que l'ouverture de session est achevée, on laisse un délai
    # de 5 secondes entre la fin du script de logon et le lancement de la
    # commande conky (avec ses arguments).
    executer_a_la_fin "5" conky --config "$REP_HOME/.conkyrc"
}
\end{lstlisting} %$
%
En principe, vous devriez voir apparaître un message incrusté
sur le bureau des utilisateurs en haut à droite. Ce message
sera légèrement personnalisé puisqu'il contiendra le nom
de l'utilisateur connecté.






\subsubsection{Exécuter des commandes au démarrage tous les 30 jours}

Toutes les commandes que vous mettrez à l'intérieur de la fonction
\verbtexte{initialisation\_perso} du fichier \verbtexte{logon\_perso} seront
exécutées à chaque phase d'initialisation du système ce qui
peut parfois s'avérer un peu trop fréquent à votre goût.
Voici un exemple de fonction \verbtexte{initialisation\_perso}
qui vous permettra d'exécuter des commandes (peu importe
lesquelles ici) au démarrage du système tous les 30 jours
(pour peu que le système ne reste pas éteint indéfiniment bien sûr) :
%
\begin{lstlisting}
function initialisation_perso ()
{
    local indicateur
    indicateur="/etc/se3/action_truc"
    # Si le fichier n'existe pas alors il faut le créer.
    [ ! -e "$indicateur" ] && touch "$indicateur"

    # On teste si la phase d'initialisation correspond à un démarrage du système.
    if "$DEMARRAGE"; then
        # On teste si la date de dernière modification du fichier est > 29 jours.
        if find "$indicateur" -mtime +29 | grep -q "^$indicateur$"; then
            echo "Les conditions sont vérifiées, on lance les actions souhaitées."
            action1
            action2
            # etc. 
            
            # Si tout s'est bien déroulé, alors on peut mettre à jour la date
            # de dernière modification du fichier avec la commande touch.
            if [ "$?" = "0"  ]; then
                touch "$indicateur"
            fi
        fi
    fi
}
\end{lstlisting}
%
L'idée de ce code est plus simple qu'il n'y paraît. Chaque client GNU/Linux
intégré au domaine possède un répertoire local \verbtexte{/etc/se3/}
(accessible en lecture et en écriture au compte \verbtexte{root}
uniquement). Dans ce répertoire, le script y place un fichier texte
vide qui se nomme \verbtexte{action\_truc} (c'est un exemple) et
dont le seul but est de fournir une date de dernière modification.
Au départ, cette date de dernière modification coïncide au
moment où le fichier est créé. Si, lors d'un prochain
démarrage, cette date de dernière modification est vieille de
30 jours ou plus, alors les actions sont exécutées et la date
de dernière modification du fichier \verbtexte{action\_truc}
est modifiée artificiellement en la date du jour avec la
commande \verbtexte{touch}.





\section{Les logs pour détecter un problème}

Après modification du script de logon, vous n'obtiendrez
peut-être pas le comportement souhaité. Peut-être parce que vous
aurez tout simplement commis des erreurs. Afin de faire un diagnostic, il
vous sera toujours possible de consulter, \textbf{sur un client
GNU/Linux}, quelques fichiers log qui se trouvent tous dans le répertoire
\verbtexte{/etc/se3/log/}. Voici la liste des fichiers log
disponibles : 


%
\begin{itemize}

\item \verbtexte{0.maj\_logon.log} : la mise à jour du script de
logon local (via son remplacement par une copie de la version distante) est
un moment important et ce fichier indiquera si cette mise à jour
a marché ou non. La date de la mise à jour y est indiquée.


\item \verbtexte{1.initialisation.log} : ce fichier contiendra
tous les messages (d'erreur ou non) suite à l'exécution du script
de logon \textbf{local} lors de la phase d'initialisation.

\item \verbtexte{1.initialisation\_distant.log} : ce fichier contiendra
tous les messages (d'erreur ou non) suite à l'exécution, lors
de la phase d'initialisation, du script
de logon \textbf{distant} (celui qui se trouve sur le serveur) et non
celui qui se trouve en local sur le client GNU/Linux.
Rappelez-vous que cela se produit quand les deux versions du script
de logon (la version locale et la version et distante) sont différentes
(ce qui est censé se produire ponctuellement seulement puisque la version locale
est ensuite mise à jour).

\item \verbtexte{1.initialisation\_perso.log} : ce fichier contiendra
tous les messages (d'erreur ou non) suite à l'exécution, lors
de la phase d'initialisation, de votre fonction \verbtexte{initialisation\_perso}.

\item \verbtexte{2.ouverture.log} : ce fichier contiendra
tous les messages (d'erreur ou non) suite à l'exécution du script
de logon local lors de la phase d'ouverture.

\item \verbtexte{2.ouverture\_perso.log} : ce fichier contiendra
tous les messages (d'erreur ou non) suite à l'exécution, lors
de la phase d'ouverture, de votre fonction \verbtexte{ouverture\_perso}.

\item \verbtexte{3.fermeture.log} : ce fichier contiendra
tous les messages (d'erreur ou non) suite à l'exécution du script
de logon local lors de la phase de fermeture.

\item \verbtexte{3.fermeture\_perso.log} : ce fichier contiendra
tous les messages (d'erreur ou non) suite à l'exécution, lors
de la phase de fermeture, de votre fonction \verbtexte{fermeture\_perso}.

\end{itemize}

À chaque fois que le script de logon s'exécute, avant d'écrire
sur le fichier \verbtexte{xxx.log} adapté à la situation du moment,
le fichier \verbtexte{xxx.log}, s'il existe déjà,
est d'abord vidé de son contenu. Donc les fichiers log ne seront
jamais très gros. Par exemple, dans le fichier 
\verbtexte{1.initialisation.log}, vous aurez des informations
portant uniquement sur la dernière phase d'initialisation effectuée par
le client GNU/Linux (pas sur les phases d'initialisation précédentes).

\section{Le cas des classes nomades}
\label{classes-nomades}

Utiliser GNU/Linux sur des ordinateurs portables dans un domaine Se3 présente un
atout extraordinaire: le mécanisme des profils (voir
section~\ref{mecanisme-profils}, page~\pageref{mecanisme-profils}) limite au
maximum les échanges entre le serveur et le client une fois la session ouverte.
Autrement dit, il n'y a aucun risque de voir la session ouverte «~planter~» en
raison d'une micro-coupure wifi.

L'intégration au domaine d'un ordinateur issu d'une classe nomade ne présente
qu'une spécificité: le client doit, avant l'ouverture de session, déjà être
connecté au réseau sans fil. Pour ce faire, il suffira d'indiquer dans le
fichier \verbtexte{/etc/network/interfaces} le SSID et le mot de passe du réseau
sans fil auquel le client est censé se connecter.. Il est également recommandé,
par la même occasion, de désactiver l'interface ethernet, sans quoi le processus
de boot se trouvera allongé de plusieurs secondes voire dizaines de secondes
(durant lesquelles le client cherchera à obtenir une IP du serveur sur toutes
les interfaces activées).

Un moyen extrêmement simple et rapide de réaliser cette manipulation est bien
sûr d'utiliser \verbtexte{unefois/}. Ainsi, admettons que les clients de votre
classe nomade soit nommés \verbtexte{nomade-01}, \verbtexte{nomade-02}, jusqu'à
\verbtexte{nomade-15}. Avant leur intégration au domaine, vous pouvez par
exemple:
\begin{itemize}
 \item déposer dans le répertoire \verbtexte{/var/www/} de votre Se3 le fichier
\verbtexte{interfaces} configuré par vos soins
 \item préparer un script intitulé \verbtexte{reseau-nomade.unefois} contenant
les lignes suivantes\footnote{Changez bien sûr \verbtexte{IP\_DE\_VOTRE\_SE3}...
par l'IP réelle de votre Se3...} :

\begin{lstlisting}
#!/bin/bash
wget http://IP_DE_VOTRE_SE3/interfaces
mv /etc/network/interfaces /etc/network/interfaces.old
mv interfaces /etc/network/
/etc/init.d networking restart
\end{lstlisting}

 \item déposer dans le répertoire
\verbtexte{/home/netlogon/clients-linux/unefois/\string^nomade-/} du serveur ledit
script
 \item lancer l'intégration au domaine de vos clients
\end{itemize}

Les ordinateurs de la classe nomade redémarreront une première fois après
l'intégration au domaine: laissez les branchés en filaire. Lors de leur premier
boot en version «~intégrés~», les clients récupéreront le fichier de
configuration du réseau et se connecteront automatiquement au réseau wifi
adéquat.

Le jour où vous aurez besoin de faire d'importantes mises à jour, vous pourrez
tout aussi facilement refaire momentanément basculer ces postes en filaire...


\section{Un mot sur les imprimantes}
\label{imprimante}

Ne disposant personnellement d'aucune imprimante réseau, je n'ai jamais pu tester
ce qui suit%
\footnote{Si vous avez du code bash à me proposer pour automatiser
l'installation des imprimantes sur les clients GNU/Linux via par exemple la fonction
\verbtexte{initialisation\_perso}, je suis preneur
(\email{francois.lafont@crdp.ac-versailles.fr}).}%
%
.
Je suis donc loin de maîtriser l'aspect \og gestion des imprimantes \fg{} sur les
clients GNU/Linux. Ceci étant, il faut bien évoquer ce point très important.

Sur un client GNU/Linux, le répertoire \verbtexte{/mnt/netlogon/divers/}
contient un sous-répertoire nommé \verbtexte{imprimantes/}  qui vous permettra
de stocker de manière centralisée des fichiers \verbtexte{.ppd}
(pour \og PostScript Printer Description \fg) qui sont des sortes
de drivers permettant d'installer
des imprimantes sur les clients GNU/Linux. Vous pouvez télécharger de tels
fichiers (qui dépendent du modèle de l'imprimante) sur ce site par exemple :
%
\begin{center}
\url{http://www.openprinting.org/printers}
\end{center}
%
Supposons que, dans le répertoire \verbtexte{/mnt/netlogon/divers/imprimantes/},
se trouve  le fichier \verbtexte{.ppd} d'un modèle d'imprimante réseau donné,
vous pouvez alors
lancer son installation sur un client GNU/Linux via la commande suivante
(en tant que \verbtexte{root}) :
%
\begin{lstlisting}[emph={ENTREE},emphstyle={\return}]
lpadmin -p NOM-IMPRIMANTE -v socket://IP-IMPRIMANTE:9100 \ENTREE
    -E -P /mnt/netlogon/divers/imprimantes/fichier.ppd
\end{lstlisting}
%
Cette commande doit être en principe exécutée une seule fois sur le client
GNU/Linux. Si tout va bien, vous devriez ensuite%
%
\footnote{Même après redémarrage du système.}
%
être en mesure d'imprimer
tout ce que vous souhaitez à travers vos applications favorites
(navigateur Web, traitement de texte, lecteur de PDF etc).
Si plusieurs imprimantes sont installées sur un client, 
pour faire en sorte que l'imprimante \verbtexte{NOM-IMPRIMANTE}
soit l'imprimante par défaut, il faut exécuter en tant que \verbtexte{root} :
%
\begin{lstlisting}
lpadmin -d NOM-IMPRIMANTE 
\end{lstlisting}
%
Et pour supprimer l'imprimante :
%
\begin{lstlisting}
lpadmin -x NOM-IMPRIMANTE 
\end{lstlisting}






\section{Désinstallation/réinstallation du paquet se3-clients-linux}
\label{desinstallation}


\subsection{Désinstallation complète}

Si jamais vous souhaitez désinstaller complètement le paquet \verbtexte{se3-clients-linux}
de votre serveur, rien de plus simple. En tant que \verbtexte{root} sur le serveur,
il suffit de lancer la commande suivante :
%
\begin{lstlisting}
apt-get purge se3-clients-linux
\end{lstlisting}
%
Et c'est tout. Une fois la commande ci-dessus exécutée,
votre serveur ne garde \textbf{plus la moindre trace} d'installation du
paquet \verbtexte{se3-clients-linux}.

\begin{alerte}
Attention, en désinstallant le paquet de la sorte (avec \verbtexte{apt-get purge}),
tout le répertoire \verbtexte{/home/netlogon/clients-linux/}
du serveur (et tout ce qu'il contient) sera effacé.
Si vous aviez pris la peine de vous concocter un fichier
\verbtexte{logon\_perso} à votre sauce, d'écrire de nombreux
scripts dans le répertoire \verbtexte{unefois/} etc., tout sera
purement et simplement effacé.
\end{alerte}

\subsection{Désinstallation partielle en vue d'une réinstallation}
\label{reinstallation}


Avec la commande ci-dessous (où l'instruction \verbtexte{purge}
est remplacée par \verbtexte{remove}), les choses se passent un peu différemment :
%
\begin{lstlisting}
apt-get remove se3-clients-linux
\end{lstlisting}
%
Le paquet \verbtexte{se3-clients-linux} est bien désinstallé
comme dans le cas précédent, sauf que le répertoire 
\verbtexte{/home/netlogon/clients-linux/} du serveur n'est
pas totalement effacé. Tous les fichiers/répertoires que
vous avez le droit de modifier seront conservés, si
bien que l'arborescence du répertoire ressemblera à ceci :
%
\begin{lstlisting}[emph={logon_perso,skel,unefois,divers},emphstyle={\color{green}\textbf}]
-- clients-linux/
   |-- bin/
   |   `-- logon_perso
   |-- distribs/
   |   |-- precise/
   |   |   `-- skel/
   |   `-- squeeze/
   |       `-- skel/
   |-- divers/
   `-- unefois/
\end{lstlisting}
%
Ainsi, après réinstallation du paquet, vous retrouverez inchangés :
%
\begin{itemize}
\item le fichier \verbtexte{logon\_perso} ;
\item tous les profils distants de chaque distribution 
prise en charge ;
\item le contenu du répertoire \verbtexte{divers/} ;
\item le contenu du répertoire \verbtexte{unefois/}.
\end{itemize}
%
En résumé, une réinstallation du paquet \verbtexte{se3-clients-linux}
avec conservation des fichiers/dossiers modifiables se fait ainsi :
%
\begin{lstlisting}
apt-get remove se3-clients-linux
apt-get install se3-clients-linux
\end{lstlisting}
%
Une telle réinstallation du paquet peut être utile si jamais,
pour une raison ou pour une autre,
vous avez commis un certain nombre de modifications malheureuses 
en voulant \og hacker \fg{} certains fichiers du paquet et que vous souhaitez
repartir de zéro sans pour autant perdre vos fichiers \og personnels%
%
\footnote{Ce qu'on appelle les fichiers \og personnels \fg{}, 
ce sont les fichiers que vous avez le droit de modifier, ceux qui
sont en vert dans l'arborescence située à la section~\ref{arborescence}
page~\pageref{arborescence}.}
%
\fg{}.
Autre cas où la réinstallation peut être utile : lors d'une mise à jour
du paquet (auquel cas d'ailleurs il sera plus naturel d'exécuter la commande
\og \verbtexte{apt-get dist-upgrade} \fg{}). Dans ce cas aussi, les
fichiers \og personnels \fg{} seront conservés en l'état.


\begin{RQ}
ici, la notion de fichiers/répertoires \og modifiables \fg{} ou
\og personnels \fg{} n'est pas à prendre
au pied de la lettre. Dans l'absolu, vous pouvez tenter de modifier ce que vous
voulez dans les fichiers du paquet \verbtexte{se3-clients-linux}.
Simplement, sont considérés comme
\og modifiables \fg{} (ou \og personnels \fg{})
seulement les fichiers/répertoires \textbf{conservés}
lors d'une réinstallation ou d'une mise à jour du paquet.
\end{RQ}


\section{Signaler un problème, faire une remarque etc.}

Étant donné que le paquet \verbtexte{se3-clients-linux} fait partie du
projet SambaÉdu, le mieux à faire pour signaler un problème, faire
une remarque etc. est de passer par la liste de diffusion SambaÉdu
\email{samba-edu@mailinglistes.tice.ac-caen.fr}%
%
\footnote{Attention, pour pouvoir écrire à cette liste de diffusion,
il faut d'abord s'y inscrire :
\url{http://listes.tice.ac-caen.fr/mailman/listinfo/samba-edu}.}.
%
N'hésitez pas à me signaler toute erreur.
Si vous envoyez un message sur cette liste avec un objet assez évocateur (par exemple
avec l'expression \og clients GNU/Linux \fg{} dedans), il y a peu de chance
que je passe à côté. J'essayerai alors de vous répondre et, dans la mesure
du possible, de rectifier le problème.



\section{Contribuer à améliorer le paquet}

Dans votre coin, vous avez réussi à modifier le paquet (un script d'intégration,
le script de logon etc.) afin d'en étendre les fonctionnalités
(par exemple afin de prendre en charge une nouvelle distribution, un
autre gestionnaire de bureau qui a votre faveur et qui n'est pas actuellement
pris en charge par le paquet etc. etc. etc.) ? Si c'est le cas, n'hésitez surtout
pas à nous en faire part sur la liste de diffusion
\email{se3-devel@listes.tice.ac-caen.fr}. Nous serons ravis d'intégrer au paquet
vos contributions pour peu que vous les ayez déjà testées avant%
%
\footnote{Car tester une fonctionnalité prend du temps (la coder aussi d'ailleurs),
et c'est le temps qui nous limite dans l'élaboration du paquet et
non un manque de volonté bien sûr.}.





\section{Les évolutions du paquet}

Ci-dessous, vous trouverez le contenu du fichier \verbtexte{changelog}
du paquet. C'est une sorte de journal qui décrit les modifications
du paquet au fur et à mesure du temps :

\begin{center}
\noindent\hrulefill\par
\verbatiminput{../se3-clients-linux/DEBIAN/changelog}
\noindent\hrulefill
\end{center}
